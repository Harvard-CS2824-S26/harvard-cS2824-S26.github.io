\chapter{Optimality}

We now seek to understand the global convergence properties of policy
gradient methods, when given exact gradients. Here, we will largely limit
ourselves to the (tabular) softmax policy class in Example~\ref{ex:softmax}.

Given our a starting distribution $\rho$ over states, recall our
objective is:
\[
\max_{\theta \in \Theta}  V^{\pi_\theta}(\rho) \, .
\]
where $\{\pi_\theta| \theta \in \Theta \subset \R^d\}$
is some class of parametric policies.

While we are interested in good performance
under $\rho$, we will see how it is helpful to optimize under
a different measure $\mu$. Specifically, we consider optimizing
$V^{\pi_\theta}(\mu)$, i.e.
\[
\max_{\theta \in \Theta}  V^{\pi_\theta}(\mu) \, ,
\]
even though our ultimate goal is performance
under $V^{\pi_\theta}(\rho)$.

We now consider the softmax policy parameterization~\eqref{eq:softmax}.
Here, we still have a
non-concave optimization problem in general, as shown in Lemma
\ref{lemma:softmax-noncon}, though we do show that global optimality can
be reached under certain regularity conditions. From a practical perspective, the softmax parameterization
of policies is preferable to the direct
parameterization, since the parameters $\theta$ are unconstrained
and standard unconstrained optimization algorithms can be employed.
However, optimization
over this policy class creates other challenges as we study in this
section, as the optimal policy (which is deterministic) is attained by
sending the parameters to infinity.

This chapter will study three algorithms for this
problem, for the softmax policy class. The first performs direct policy gradient ascent on the
objective without modification, while the second adds a log barrier
regularizer to keep the parameters from becoming too large, as a
means to ensure adequate exploration.
%We first show that
%gradient ascent converges in the limit; furthermore, provided we also use a
%cross-entropy regularization penalty term, we then provide a
%convergence rate.
Finally, we study the natural policy gradient
algorithm and establish a global optimality convergence rate,
with no dependence on the dimension-dependent factors.

The presentation in this chapter largely follows the results
in~\cite{agarwal2019optimality}. 


\begin{figure}
		\centering

			\begin{tikzpicture}[->,>=stealth,node distance=2.0cm]
		\node[state] (s0) {$s_0$};
		\node[state, right of=s0] (s1) {$s_1$};
		\node[state, draw=none] (d1) [right=of s1, xshift=-2.0cm] {$\cdots$};
		\node[state, right of=d1] (sh) {$s_{H}$};
		\node[state, accepting, right of=sh] (shp) {$s_{H+1}$};
		\draw (shp) edge[loop above] node{$a_1$} (shp)
		(s0) edge[bend right=30, below] node{$a_1$} (s1)
		(d1) edge[bend right=30, below] node{$a_1$} (sh)
		(sh) edge[bend right=0, below] node{$a_1$} (shp)
		(s1) edge[bend right=0, above] node{$a_2$} (s0)
		(sh) edge[bend right=0, above] node{$a_2$} (d1)
		(s1) edge[bend right=40, above] node{$a_3$} (s0)
		(s1) edge[bend right=90, above] node{$a_4$} (s0)
		(sh) edge[bend right=40, above] node{$a_3$} (d1)
		(sh) edge[bend right=90, above] node{$a_4$} (d1);
			\end{tikzpicture}
			\caption{(Vanishing gradient example) A deterministic, chain MDP of length $H+2$. We
                          consider a policy where $\pi(a | s_i) =
                          \theta_{s_i,a}$ for $i=1,2,\ldots,H$. Rewards
                          are $0$ everywhere other than $r(s_{H+1},
                          a_1) = 1$. See Proposition~\ref{proposition:small_grad}.}
			\label{fig:chain}

	\end{figure}


\section{Vanishing Gradients and Saddle Points}
\label{sec:vanishing-gradients}

To understand the necessity of optimizing under a distribution $\mu$
that is different from $\rho$, let us first give an informal argument that
some condition on the state distribution of $\pi$, or equivalently
$\mu$, is necessary for stationarity to imply optimality. For example,
in a sparse-reward MDP (where the agent is only rewarded upon visiting
some small set of states), a policy that does not visit \emph{any}
rewarding states will have zero gradient, even though it is
arbitrarily suboptimal in terms of values. Below, we give a more
quantitative version of this intuition, which demonstrates that even
if $\pi$ chooses all actions with reasonable probabilities (and hence
the agent will visit all states if the MDP is connected), then there
is an MDP where a large fraction of the policies $\pi$ have vanishingly small gradients, and yet
these policies are highly suboptimal in terms of their value.

Concretely, consider the chain MDP of length $H+2$ shown in
Figure~\ref{fig:chain}. The starting state of interest is state $s_0$
and the discount factor $\gamma = H/(H+1)$.
Suppose we work with the direct parameterization, where $\pi_\theta(a | s) =
\theta_{s,a}$ for $a = a_1,a_2,a_3$ and $\pi_\theta(a_4 | s) = 1 -
\theta_{s,a_1} - \theta_{s,a_2} - \theta_{s,a_3}$. Note we do not
over-parameterize the policy.
%and, instead, enforce the normalization explicitly
%in the parameterization.
For this MDP and policy structure, if we were to initialize the
probabilities over actions, say deterministically, then there is an
MDP (obtained by permuting the actions) where all the probabilities
for $a_1$ will be less than $1/4$.

The following result not only shows that the gradient is exponentially
small in $H$, it also shows that many higher order
derivatives, up to $O(H/\log H)$, are also exponentially
small in $H$.

\begin{proposition} [Vanishing gradients at suboptimal parameters]
\label{proposition:small_grad}
Consider the chain MDP of Figure~\ref{fig:chain}, with $H+2$ states, $\gamma =
H/(H+1)$, and with the direct
policy parameterization (with $3|\Scal|$ parameters, as described in the
text above).
Suppose $\theta$ is such that $0<\theta<1$ (componentwise) and
$\theta_{s,a_1} < 1/4$ (for all states $s$). For all
$k \leq  \frac{H}{40\log(2 H)} - 1$, we have $\norm{\nabla_\theta^k V^{\pi_\theta}(s_0)} \leq
(1/3)^{H/4}$, where $\nabla_\theta^k V^{\pi_\theta}(s_0)$ is a tensor of the $k_{th}$
order derivatives of $V^{\pi_\theta}(s_0)$ and the norm is the operator norm of the
tensor.\footnote{The operator norm of a $k_{th}$-order tensor $J \in
  \R^{d^{\otimes k}}$ is defined as $\sup_{u_1,\ldots,u_k \in
    \R^d~:~\|u_i\|_2 = 1} \langle J, u_1\otimes\ldots\otimes
  u_d\rangle$.}  Furthermore, \mbox{$V^{\star}(s_0) - V^{\pi_\theta}(s_0) \geq
(H+1)/8-(H+1)^2/3^H$}.
\end{proposition}

We do not prove this lemma here (see Section~\ref{bib:pg_opt}).
The lemma illustrates that lack of good exploration can indeed be
detrimental in policy gradient algorithms, since the gradient can be
small either due to $\pi$ being near-optimal, or, simply because $\pi$ does
not visit advantageous states often enough. Furthermore, this lemma also suggests that varied results in the
non-convex optimization literature, on escaping from saddle
points, do not directly imply global convergence due to that the higher order
derivatives are small.

While the chain MDP of Figure~\ref{fig:chain},
is a common example where \emph{sample} based estimates of gradients
will be $0$ under random exploration strategies; there is an
exponentially small in $H$  chance of hitting the goal state under a
random exploration strategy. Note
that this lemma is with regards to \emph{exact} gradients. This
suggests that even with exact computations (along with using exact higher
order derivatives) we might expect numerical instabilities.


\section{Policy Gradient Ascent}

Let us now return to the softmax policy class, from
Equation~\ref{eq:softmax}, where:
\[
\pi_\theta(a|s) = \frac{\exp(\theta_{s,a})}{\sum_{a'} \exp(\theta_{s,a'})}
\, ,
\]
where the number of parameters in this policy class is $|\Scal| |\Acal|$.

Observe that:
\[
\frac{\partial \log \pi_\theta(a|s)}{\partial \theta_{s',a'}} =
\mathds{1} \Big[ s=s' \Big]\Big(\mathds{1} \Big[ a=a' \Big] - \pi_\theta(a'|s)\Big)
\]
where $\mathds{1}[\mathcal{E}]$ is the indicator of $\mathcal{E}$ being
true.

\begin{lemma}
For the softmax policy class, we have:
\begin{equation} \label{eq:softmax-grad}
  \frac{\partial V^{\pi_\theta}(\mu)}{\partial \theta_{s,a}} =
\frac{1}{1-\gamma} d_\mu^{\pi_\theta}(s)\pi_\theta(a|s)A^{\pi_\theta}(s,a)
\end{equation}
\end{lemma}

\begin{proof}
Using the advantage expression for the policy gradient (see Theorem~\ref{thm:pg_expressions}),
  \begin{eqnarray*}
          \frac{\partial V^{\pi_\theta}(\mu)}{\partial \theta_{s',a'}}    &=& 
\frac{1}{1-\gamma}\E_{s\sim d^{\pi_\theta}_\mu }\E_{a\sim \pi_\theta(\cdot|s) }
\Big[A^{\pi_\theta}(s,a) \frac{\partial \log  \pi_\theta(a|s)}{\partial \theta_{s',a'}}\Big]\\
 &=& 
\frac{1}{1-\gamma}\E_{s\sim d^{\pi_\theta}_\mu }\E_{a\sim \pi_\theta(\cdot|s) }
\Big[A^{\pi_\theta}(s,a)
\mathds{1} \Big[ s=s' \Big]\Big(\mathds{1} \Big[ a=a' \Big] - \pi_\theta(a'|s)\Big)\Big]\\
 &=& 
\frac{1}{1-\gamma}d _\mu^{\pi_\theta} (s')\E_{a\sim \pi_\theta(\cdot|s') }
\Big[A^{\pi_\theta}(s',a)\Big(\mathds{1} \Big[ a=a' \Big] - \pi_\theta(a'|s')\Big)\Big]\\
 &=& 
\frac{1}{1-\gamma}d _\mu^{\pi_\theta} (s')\left(\E_{a\sim \pi_\theta(\cdot|s') }
\Big[A^{\pi_\theta}(s',a)\mathds{1} \big[ a=a' \big] \Big]
-\pi_\theta(a'|s')
\E_{a\sim \pi_\theta(\cdot|s') }\Big[A^{\pi_\theta}(s',a)\Big]\right)\\
&=&
\frac{1}{1-\gamma}d _\mu^{\pi_\theta} (s') \pi_\theta(a'|s') A^{\pi_\theta}(s',a') -0\, ,
      \end{eqnarray*}
where the last step uses that for any policy $\sum_a \pi(a|s) A^\pi(s,a)=0$. 
\end{proof}

%\subsubsection*{Global Convergence}

The update rule for gradient ascent is:
\begin{equation}    \label{eqn:gd-softmax}
    \theta^{(t+1)} = \theta^{(t)} + \eta \nabla_\theta V^{(t)}(\mu).
\end{equation}

Recall from Lemma~\ref{lemma:softmax-noncon} that, even for the case of the softmax policy class (which
contains all stationary policies), our optimization problem is
non-convex. Furthermore, due to the exponential scaling with the
parameters $\theta$ in the softmax parameterization, \emph{any} policy
that is nearly deterministic will have gradients close to $0$.
Specifically, for any sequence of policies $\pi^{\theta_t}$
that becomes deterministic, $\| \nabla
V^{\pi^{\theta_t}}\|\rightarrow 0$.

In spite of these difficulties, it turns out we have a positive result
that gradient ascent asymptotically converges to the global optimum for the softmax
parameterization.

\begin{theorem} [Global convergence for softmax parameterization]
	\label{thm:glb-softmax}
Assume we follow the gradient ascent update rule as specified
in Equation~\eqref{eqn:gd-softmax} and that the distribution
$\mu$ is strictly positive i.e. $\mu(s)>0$ for all states $s$.
Suppose $\eta \leq \frac{(1-\gamma)^3}{8}$, then we have that for all
states $s$,
$V^{(t)}(s) \rightarrow V^\star(s)$ as $t \rightarrow \infty$.
\end{theorem}

The proof is somewhat technical, and we do not provide a proof here
(see Section~\ref{bib:pg_opt}). 


A few remarks are in order. Theorem~\ref{thm:glb-softmax} assumed
that optimization distribution $\mu$ was \emph{strictly} positive,
i.e. $\mu(s)>0$ for all states $s$. We conjecture that any 
gradient ascent may not globally converge if this condition is not met. 
The concern is that if this condition is not met, then gradient ascent
may not globally converge due to that $d^{\pi_\theta}_{\mu}(s)$
effectively scales down the learning rate for the parameters
associated with state $s$ (see Equation~\ref{eq:softmax-grad}).
%\end{remark}

Furthermore, there is strong reason to believe that the convergence
rate for this is algorithm (in the worst case) is exponentially slow in some of the
relevant quantities, such as in terms of the size of state space. We now turn to a regularization based approach
to ensure convergence at a polynomial rate in all relevant quantities.

\iffalse
The complete proof is provided in the Appendix~\ref{app:gd-softmax}.  We
now discuss the subtleties in the proof and show why the softmax
parameterization precludes a direct application of the gradient
domination lemma. In order to utilize the gradient domination
property (in Lemma~\ref{thm:first}), we would desire to show that:
$ \nabla_\pi V^{\pi}(\mu) \rightarrow 0 $. However, using the
functional form of the softmax parameterization (see Equation
\ref{eq:softmax-grad}) and \eqref{eq:grad_direct}, we have that:
\begin{equation*}%\label{eq:softmax_grad}
\frac{\partial V^{\pi_\theta}(\mu)}{\partial \theta_{s,a}} =
\frac{1}{1-\gamma} d^{\pi_\theta}_{\mu}(s)\pi_{\theta}(a|s)A^{\pi_\theta}(s,a)
= \pi_\theta(a|s) \, \frac{\partial V^{\pi_\theta}(\mu)}{\partial \pi_\theta(a|s)} .
\end{equation*}
Hence, we see that
even if $\nabla_\theta V^{\pi_\theta}(\mu) \rightarrow 0$, we are not
guaranteed that $\nabla_\pi V^{\pi_\theta}(\mu) \rightarrow 0$.

We now briefly discuss the main technical challenges in the proof. The
proof first
shows that the sequence
$V^{(t)}(s)$ is monotone increasing pointwise, i.e. for \emph{every} state $s$, $V^{(t+1)}(s) \geq V^{(t)}(s)$
(Lemma \ref{lem:q-improv}). This implies the existence of a limit $V^{(\infty)}(s)$
by the monotone convergence theorem (Lemma \ref{lem:delta}). Based on
the limiting quantities $V^{(\infty)}(s)$
and $Q^{(\infty)}(s,a)$, which we show exist, define the following limiting sets for each state $s$:
\begin{eqnarray*}
		I^s_{0} &:=&\{a | Q^{(\infty)}(s,a) = V^{(\infty)}(s)\}\\
		I^s_{+} &:=&\{a | Q^{(\infty)}(s,a) > V^{(\infty)}(s)\}\\
		I^s_{-} &:=&\{a | Q^{(\infty)}(s,a) < V^{(\infty)}(s)\} \, .
\end{eqnarray*}
The challenge is to then show that, for all states $s$, the
set $I^s_{+} $ is the empty set, which would immediately imply
$V^{(\infty)}(s) = V^\star(s)$.  The proof proceeds by contradiction,
assuming that $I^s_{+} $ is non-empty. Using that $I^s_{+} $ is
non-empty and that the gradient tends to zero in the limit, i.e.
$\nabla_\theta V^{\pi_\theta}(\mu) \rightarrow 0$, we have
that for all $a \in I^s_{+}$, $\pi^{(t)}(a|s) \rightarrow 0$ (see
\eqref{eq:softmax_grad}).  This, along with the functional form of the
softmax parameterization, implies that there must be divergence (in
magnitude) among the set of parameters associated with \emph{some}
action $a$ at state $s$, i.e. that
$\max_{a\in \Acal} |\theta^{(t)}_{s,a}| \to \infty$. The primary
technical challenge in the proof is to then use this divergence, along
with the dynamics of gradient ascent, to show that $I^s_{+} $ is empty
via a contradiction.
\fi



\section{Log Barrier Regularization}

Due to the exponential scaling with the parameters $\theta$, policies
can rapidly become near deterministic, when optimizing under the softmax
parameterization, which can result in slow convergence. Indeed a key
challenge in the asymptotic analysis in the previous section was to
handle the growth of the absolute values of parameters as they tend to
infinity. 
Recall that the relative-entropy for
distributions $p$ and $q$ is defined as:
\[
\textrm{KL}(p,q) := \E_{x\sim p}[-\log q(x)/p(x)].
\]

Denote the uniform distribution over a set $\mathcal{X}$ by
$\textrm{Unif}_\mathcal{X}$, and define the  following log barrier
regularized objective as:
\begin{eqnarray}
L_\lambda(\theta) &:=& V^{\pi_\theta}(\mu) - \lambda \,
\E_{s\sim \textrm{Unif}_\Scal} \bigg[\textrm{KL} (\textrm{Unif}_\Acal,\pi_\theta(\cdot|s))
\bigg]\nonumber\\
&  = &  V^{\pi_\theta}(\mu) + \frac{\lambda}{|\Scal|\, |\Acal|} \sum_{s,a} \log \pi_\theta(a|s)
+\lambda \log |\Acal| \, , \label{eqn:loss-reg}
\end{eqnarray}
where $\lambda$ is a regularization parameter. The constant (i.e. the
last term) is
not relevant with regards to optimization.
This
regularizer is different from the more commonly utilized entropy
regularizer, a point which we return to later.

\iffalse
A
common practical remedy for this is to use entropy-based
regularization to keep the probabilities from getting too
small~\citep{williams1991function, mnih2016asynchronous}, and we study
gradient ascent on a similarly regularized objective in this
section.
\fi


The policy gradient ascent updates for
$L_\lambda(\theta)$ are given by:
\begin{equation}
    \theta^{(t+1)} = \theta^{(t)} + \eta \nabla_\theta L_\lambda(\theta^{(t)}).
    \label{eqn:gd-entropic}
\end{equation}
We now see that any approximate first-order stationary
points of the entropy-regularized objective is
approximately globally optimal, provided the regularization is
sufficiently small.

\begin{theorem} \label{thm:small-gradient}
(Log barrier regularization)
Suppose $\theta$ is such that:
\[
  \|\nabla_\theta L_\lambda(\theta)\|_2 \leq \epsopt
\]
and $\epsopt \leq \lambda/(2|\Scal|\, |\Acal|)$. Then we have
that for all starting state distributions $\rho$:
\begin{eqnarray*}
  V^{\pi_\theta}(\rho)
  &\geq& V^\star (\rho) -
         \frac{2 \lambda}{1-\gamma} \Norm{\frac{d^{\pi^\star}_{\rho}}{\mu } }_\infty         .
\end{eqnarray*}
%\label{lem:small-gradient}
\end{theorem}

We refer to $\Norm{\frac{d^{\pi^\star}_{\rho}}{\mu} }_\infty$ as the
\emph{distribution mismatch coefficient}. The above theorem shows the
importance of having an appropriate measure $\mu(s)$ in order for the
approximate first-order stationary points to be near optimal.


\begin{proof}
  The proof consists of showing that
  $\max_{a}A^{\pi_\theta}(s,a) \le 2\lambda/(\mu(s)|\Scal|)$ for all
  states. To see that this is sufficient, observe that by the
  performance difference lemma (Lemma~\ref{lemma:perf_diff}),
\begin{align*}
V^\star(\rho) -V^{\pi_\theta}(\rho)
&= \frac{1}{1-\gamma} \sum_{s,a} d^{\pi^\star}_{\rho}(s) \pi^\star (a|s) A^{\pi_\theta}(s,a) \\
&\le \frac{1}{1-\gamma} \sum_{s} d^{\pi^\star}_{\rho}(s) \max_{a\in\Acal}A^{\pi_\theta}(s,a)\\
&\leq
\frac{1}{1-\gamma} \sum_{s} 2 d^{\pi^\star}_{\rho} (s) \lambda/(\mu(s) |\Scal|)\\&
\leq
\frac{2 \lambda}{1-\gamma} \max_s\left( \frac{d^{\pi^\star}_{\rho}(s)}{\mu (s)} \right).
\end{align*}
which would then complete the proof.

We now proceed to show that $\max_{a}A^{\pi_\theta}(s,a) \le 2\lambda/(\mu(s)|\Scal|)$. For this, it
suffices to bound $A^{\pi_\theta}(s,a)$ for any state-action pair
$s,a$ where $A^{\pi_\theta}(s,a)\geq 0$ else the claim is trivially
true.  Consider an $(s,a)$ pair such that $A^{\pi_\theta}(s,a)>0 $.
%We drop the $\mu$ dependence on $d_{\mu}^{\pi_\theta}(s)$ and write
%$d^{\pi_\theta}(s)$ when clear from context.
\iffalse
Due to that
\[
L_\lambda(\theta) =V^{\pi_\theta}(\mu)+
\frac{\lambda}{|\Scal|\,|\Acal|} \sum_{s,a} \left(\theta_{s,a}- \log\left(\sum_{a'}\exp(\theta_{s,a'})\right)\right)
\, ,
\]
\fi
Using the policy gradient expression for the softmax parameterization
(see Equation~\ref{eq:softmax-grad}),
\begin{equation}\label{eqn:grad-KLobjective}
\frac{\partial L_\lambda(\theta)}{\partial \theta_{s,a}}
= \frac{1}{1-\gamma}
d^{\pi_\theta}_\mu(s) \pi_\theta(a|s)A^{\pi_\theta}(s,a)
+ \frac{\lambda}{|\Scal|} \left(\frac{1}{|\Acal|}-\pi_\theta(a|s)\right)
\, .
\end{equation}
The gradient norm assumption $\|\nabla_\theta L_\lambda(\theta)\|_2 \leq \epsopt$ implies that:
\begin{align*}
\epsopt \geq
\frac{\partial L_\lambda(\theta)}{\partial \theta_{s,a}}
&= \frac{1}{1-\gamma}
d^{\pi_\theta}_\mu(s)\pi_\theta(a|s)A^{\pi_\theta}(s,a)
+ \frac{\lambda}{|\Scal|} \left(\frac{1}{|\Acal|}-\pi_\theta(a|s)\right)\\&
\geq  \frac{\lambda}{|\Scal|} \left(\frac{1}{|\Acal|}-\pi_\theta(a|s)\right),
\end{align*}
where we have used $A^{\pi_\theta}(s,a)\geq 0$. Rearranging and using our assumption $\epsopt \leq \lambda/(2 |\Scal|\,|\Acal|)$,
\[
\pi_\theta(a|s) \geq \frac{1}{|\Acal|} - \frac{\epsopt|\Scal|}{\lambda} \geq \frac{1}{2|\Acal|} \, .
\]
Solving for $A^{\pi_\theta}(s,a)$ in \eqref{eqn:grad-KLobjective}, we have:
\begin{eqnarray*}
A^{\pi_\theta}(s,a)
&=& \frac{1-\gamma}{d^{\pi_\theta}_\mu(s) }
\left(\frac{1}{\pi_\theta(a|s)} \frac{\partial L_\lambda(\theta)}{\partial \theta_{s,a}} +\frac{\lambda}{|\Scal|} \left(1-\frac{1}{\pi_\theta(a|s)|\Acal|}\right) \right)\\
&\leq& \frac{1-\gamma}{d^{\pi_\theta}_\mu(s)}\left( 2|\Acal| \epsopt+\frac{\lambda}{|\Scal|} \right) \\
&\leq& 2 \frac{1-\gamma}{d^{\pi_\theta}_\mu(s)} \frac{\lambda}{|\Scal|} \\
&\leq& 2 \lambda /(\mu(s) |\Scal|) \, ,
\end{eqnarray*}
where the penultimate step uses $\epsopt \leq \lambda/(2 |\Scal|\,|\Acal|)$ and the
final step uses $d^{\pi_\theta}_{\mu}(s)\geq
(1-\gamma)\mu(s)$. This completes the proof.
\end{proof}

The policy gradient ascent updates for
$L_\lambda(\theta)$ are given by: 
\begin{equation}
    \theta^{(t+1)} = \theta^{(t)} + \eta \nabla_\theta L_\lambda(\theta^{(t)}).
    \label{eqn:gd-entropic}
\end{equation}


By combining the above theorem with the
convergence of gradient ascent to first order stationary points
(Lemma~\ref{lemma:stationary}), we obtain the following corollary. 
\begin{corollary}
(Iteration complexity with log barrier regularization)
Let $\beta_\lambda :=\frac{8\gamma }{(1-\gamma)^3} + \frac{2\lambda
}{|\Scal|}$.
  Starting from any initial
  $\theta^{(0)}$, consider the updates~\eqref{eqn:gd-entropic} with
  \mbox{$\lambda = \frac{\epsilon(1-\gamma)}{2\Norm{\frac{d^{\pi^\star}_{\rho}}{\mu} }_\infty}$} and $\eta = 1/\beta_\lambda$.
Then for all starting state distributions $\rho$, we have
    \[
      \min_{t< T} \left\{ V^\star(\rho) - V^{(t)}(\rho)\right\}
      \leq \epsilon \quad \mbox{whenever}
      \quad T \geq \frac{320 |\Scal|^2|\Acal|^2}{(1-\gamma)^6\,\epsilon^2} \Norm{\frac{d^{\pi^\star}_{\rho}}{\mu } }_\infty^2.
    \]
\label{corollary:entropy}
\end{corollary}

The corollary shows the
importance of  balancing how the regularization parameter $\lambda$
is set relative to the desired
accuracy $\epsilon$, as well as the importance of the initial
distribution $\mu$ to obtain global optimality.

\begin{proof}[\textbf{of Corollary~\ref{corollary:entropy}}]
Let $\beta_\lambda$ be the smoothness of
$L_\lambda(\theta)$. A valid upper bound on $\beta_\lambda$ is:
\[
\beta_\lambda = \frac{8\gamma }{(1-\gamma)^3} + \frac{2\lambda
}{|\Scal|} \, ,
\]
where we leave the proof as an exercise to the reader.

Using Theorem~\ref{thm:small-gradient}, the desired optimality gap
$\epsilon$ will follow if we set 
$
\lambda = \frac{\epsilon(1-\gamma)}{2\Norm{\frac{d^{\pi^\star}_{\rho}}{\mu} }_\infty}
$
and if $\|\nabla_\theta L_\lambda(\theta)\|_2 \leq \lambda/(2|\Scal|\, |\Acal|)$.
In order to complete the proof, we need to bound the iteration
complexity of making the gradient sufficiently small.

By Lemma~\ref{lemma:stationary}, after $T$ iterations of gradient ascent with stepsize of
$1/\beta_\lambda$, we have
\begin{equation}
\min_{t \leq T} \norm{\nabla_\theta L_\lambda(\theta^{(t)})}_2^2 \leq \frac{2\beta_\lambda (L_\lambda(\theta^{\star}) - L_\lambda(\theta^{(0)}))}{T} \leq \frac{2\beta_\lambda}{(1-\gamma)\, T},
\end{equation}
%
where $\beta_\lambda$ is an upper bound on the smoothness of
$L_\lambda(\theta)$.
We seek to ensure
\begin{align*}
\epsopt \leq \sqrt{\frac{2\beta_\lambda}{(1-\gamma)\, T}} \leq \frac{\lambda}{2 |\Scal|\,|\Acal|}
\end{align*}

Choosing $T \geq \frac{8\beta_\lambda\,
	|\Scal|^2|\Acal|^2}{(1-\gamma)\, \lambda^2}$ satisfies  the
above inequality. Hence,
\begin{eqnarray*}
	\frac{8\beta_\lambda\,
		|\Scal|^2|\Acal|^2}{(1-\gamma)\, \lambda^2}&\leq& \frac{64\,
		|\Scal|^2|\Acal|^2}{(1-\gamma)^4\, \lambda^2}
	+\frac{16\, |\Scal| |\Acal|^{2}}{(1-\gamma)\, \lambda}\\
	&\leq& \frac{80\,
		|\Scal|^2|\Acal|^2}{(1-\gamma)^4\, \lambda^2}\\
	&=& \frac{320\,
		|\Scal|^2|\Acal|^2 }{(1-\gamma)^6\, \eps^2}\Norm{\frac{d^{\pi^\star}_{\rho}}{\mu} }_\infty^2
\end{eqnarray*}
where we have used that $\lambda<1$. This completes the proof.
\end{proof}

\paragraph{Entropy vs. log barrier regularization.}
A commonly considered regularizer is the entropy, where the regularizer would be:
\[
\frac{1}{|\Scal|}   \sum_s H(\pi_{\theta}(\cdot|s)) = \frac{1}{|\Scal|}  \sum_{s} \sum_a -\pi_{\theta}(a|s)
\log \pi_{\theta}(a|s).
\]
Note the entropy is far less aggressive in penalizing small
probabilities, in comparison to the log barrier, which is equivalent
to the relative entropy. In particular, the entropy regularizer is
always bounded between $0$ and $\log |\Acal|$, while the relative
entropy (against the uniform distribution over actions), is bounded
between $0$ and infinity, where it tends to infinity as probabilities
tend to $0$. Here, it can be shown that the convergence rate is
asymptotically $O(1\eps)$ (see Section~\ref{bib:pg_opt}) though
it is unlikely that the convergence rate for this method
is polynomial in other relevant quantities, including  $|\Scal|$, $|\Acal|$, $1/(1-\gamma)$, and the
  distribution mismatch coefficient.
The polynomial convergence rate using the log barrier ($\textrm{KL}$) regularizer
crucially relies on the aggressive nature in which the relative
entropy prevents small probabilities.

\section{The Natural Policy Gradient}

Observe that a policy constitutes a family of probability
distributions $\{\pi_\theta(\cdot|s) | s\in \Scal\}$. We now consider
a pre-conditioned gradient descent method based on this family of
distributions. Recall that the Fisher information matrix of a parameterized density
$p_\theta(x)$ is defined as
$
%\textrm{Fisher}(\theta) = 
\E_{x\sim p_\theta } \left[
\nabla \log p_\theta(x) \nabla \log p_\theta(x)^\top
\right]
$.
Now we let us define $\mathcal{F}^\theta_\rho$ as an (average) Fisher information matrix on
the family of distributions $\{\pi_\theta(\cdot|s) | s\in
\Scal\}$ as follows:
\[
\mathcal{F}^\theta_\rho  := 
\E_{s\sim d^{\pi_\theta}_\rho }\E_{a\sim \pi_\theta(\cdot|s) } \left[
(\nabla \log \pi_\theta(a|s)) \nabla \log \pi_\theta(a|s)^\top
\right] \, .
\]
Note that the average is under the
state-action visitation frequencies.
The NPG algorithm performs gradient updates in the geometry
induced by this matrix as follows:
\begin{align}\label{eqn:npg}
%  F_\rho(\theta) &= \E_{s \sim d^{\pi_\theta}_{\rho}}
%\E_{a\sim \pi_\theta(\cdot | s) }\Big[ \nabla_\theta \log
%\pi_\theta(a| s) \Big(\nabla_\theta \log \pi_\theta(a|
%s)\Big)^\top \Big] \nonumber\\
\theta^{(t+1)} &= \theta^{(t)} + \eta F_\rho(\theta^{(t)})^\dagger \nabla_\theta V^{(t)}(\rho),
\end{align}
where $M ^\dagger$ denotes the Moore-Penrose pseudoinverse of the
matrix $M$.  

Throughout this section, we restrict to using the initial
state distribution $\rho\in \Delta(\Scal)$ in our update rule in
\eqref{eqn:npg} (so our optimization measure $\mu$ and the performance
measure $\rho$ are identical). Also, we restrict attention to states
$s\in\Sset$ reachable from $\rho$, since, without loss of generality,
we can exclude states that are not reachable under this start state
distribution\footnote{Specifically, we restrict the MDP to the set of
  states
  $\{s\in\Sset~:~\exists \pi ~~\text{such that}~~ d^\pi_\rho(s) >
  0\}$.}.

For the softmax parameterization, this method takes a particularly
convenient form; it can be viewed as a soft policy iteration update.
% the update takes 
% (see ~\citet{Kakade01}). For completeness, we provide a proof in
%Appendix~\ref{app:npg}.

\begin{lemma}
(Softmax NPG as soft policy iteration) For the softmax parameterization~\eqref{eq:softmax}, the NPG
updates~\eqref{eqn:npg} take the form:
\[
\theta^{(t+1)} = \theta^{(t)} + \frac{\eta}{1-\gamma} A^{(t)}+\eta v \quad \mbox{and}\quad \pi^{(t+1)}(a| s) = \pi^{(t)}(a| s) \frac{\exp\big(\eta A^{(t)}(s,a)/(1-\gamma)\big)}{Z_t(s)},
\]
where $Z_t(s) = \sum_{a \in \Acal} \pi^{(t)}(a| s)  \exp\big(\eta A^{(t)}(s,a)/(1-\gamma)\big)$.
\label{lemma:npg-softmax}
Here, $v$ is only a state dependent 
offset (i.e. $v_{s,a} = c_s$ for some $c_s\in \R$ for each state $s$),
and, owing to the normalization $Z_t(s)$, $v$ has no effect on the update rule.
\end{lemma}

It is important to note that while the ascent direction was derived
using the gradient $\nabla_\theta V^{(t)}(\rho)$, which depends on
$\rho$, the NPG update rule actually has no dependence on the measure
$\rho$. Furthermore, there is no dependence on the state distribution
$d^{(t)}_{\rho}$, which is due to the pseudoinverse of
the Fisher information  cancelling out the effect of the state
distribution in NPG. 

\begin{proof}
By definition of the Moore-Penrose pseudoinverse, we have that
$(\mathcal{F}^\theta_\rho)^{\dagger}\nabla V^{\pi_\theta}(\rho)=w_\star$
if an only if $w_\star$ is the minimum norm solution of:
\[
\min_w \| \nabla V^{\pi_\theta}(\rho) - \mathcal{F}^\theta_\rho w\|^2 \, .
\]

Let us first evaluate $\mathcal{F}^\theta_\rho w$.
For the softmax policy parameterization, Lemma~\ref{eq:softmax-grad} implies:
\[
w^\top \nabla_\theta \log \pi_\theta(a|s) = w_{s,a} - \sum_{a'\in \Acal} w_{s,a'} \pi_\theta(a'|s)
:= w_{s,a} - \overline{w}_s
\]
where $\overline{w}_s$ is not a function of $a$.
This implies that:
\begin{align*}
\mathcal{F}^\theta_\rho w 
&= \E_{s\sim d^{\pi_\theta}_\rho }\E_{a\sim \pi_\theta(\cdot|s) } \left[
\nabla \log \pi_\theta(a|s) \bigg(w^\top \nabla_\theta \log \pi_\theta(a|s)\bigg)
\right] \\
&= \E_{s\sim d^{\pi_\theta}_\rho }\E_{a\sim \pi_\theta(\cdot|s) } \left[
\nabla \log \pi_\theta(a|s) \bigg(w_{s,a} - \overline{w}_s\bigg)
\right] 
= \E_{s\sim d^{\pi_\theta}_\rho }\E_{a\sim \pi_\theta(\cdot|s) } \Big[
w_{s,a} \nabla \log \pi_\theta(a|s) 
\Big] ,
\end{align*}
where the last equality uses that $\overline{w}_s$ is not a function
of $s$.
Again using the functional form of derivative of the softmax policy
parameterization, we have:
\[
\bigg[\mathcal{F}^\theta_\rho w \bigg]_{s',a'} =
d^{\pi_\theta}(s')\pi_\theta(a'|s') \bigg(w_{s',a'} - \overline{w}_{s'}\bigg).
\]
This implies:
\begin{align*}
\| \nabla V^{\pi_\theta}(\rho) - \mathcal{F}^\theta_\rho w \|^2 &=
\sum_{s,a} \left( d^{\pi_\theta}(s)\pi_\theta(a|s) \bigg(
\frac{1}{1-\gamma}A^{\pi_\theta}(s,a) - \big[\mathcal{F}^\theta_\rho w\big]_{s,a}\bigg) \right)^2\\
&=
\sum_{s,a} \left( d^{\pi_\theta}(s)\pi_\theta(a|s)  \bigg(\frac{1}{1-\gamma}A^{\pi_\theta}(s,a)  - w_{s,a} 
- \sum_{a'\in \Acal} w_{s,a'} \pi_\theta(a'|s) \bigg) \right)^2.
\end{align*}
Due to that $w=\frac{1}{1-\gamma}A^{\pi_\theta}$ leads to $0$ error,
the above implies that all $0$ error solutions are of the form
$w=\frac{1}{1-\gamma}A^{\pi_\theta}+v$, where $v$ is only a state dependent 
offset (i.e. $v_{s,a} = c_s$ for some $c_s\in \R$ for each state $s$).
The first claim follows due to that the minimum norm solution is one
of these solutions. The proof of the second
claim now follows by the definition of the NPG update rule, along with
that $v$ has no effect on the update rule due to the normalization constant $Z_t(s)$.
\end{proof}

We now see that this algorithm enjoys a dimension free convergence rate.

\begin{theorem}[Global convergence for NPG]
Suppose we run the NPG updates~\eqref{eqn:npg} using $\rho\in
\Delta(\Scal)$ and with
$\theta^{(0)}=0$. Fix $\eta>0$. For all $T>0$, we have:
\[
V^{(T)}(\rho) \geq V^\ast(\rho) - \frac{\log |\Acal|}{\eta T} - \frac{1}{(1-\gamma)^2T}.
\]
\label{thm:npg}
\end{theorem}

Note in the above the theorem that the NPG algorithm is directly applied to the performance
measure $V^{\pi}(\rho)$, and the guarantees are also with respect to
$\rho$. In particular, there is no distribution mismatch coefficient
in the rate of convergence.

Now setting $\eta \geq (1-\gamma)^2 \log |\Acal|$, we see that
NPG finds an $\epsilon$-optimal policy in a number of iterations that is at most:
\[
T \leq \frac{2}{(1-\gamma)^2\epsilon} ,
\]
which has no dependence on the number of states or actions, despite the
non-concavity of the underlying optimization problem.

The proof strategy we take borrows ideas from the classical
multiplicative weights algorithm (see Section!\ref{bib:pg_opt}).

First, the following improvement lemma is helpful:

\begin{lemma}[Improvement lower bound for NPG]
	\label{lemma:npg-gap}
For the iterates $\pi^{(t)}$ generated by the NPG
updates~\eqref{eqn:npg}, we have for all starting state distributions $\mu$
\[
  V^{(t+1)}(\mu) - V^{(t)}(\mu) \geq \frac{(1-\gamma)}{\eta} \E_{s\sim
    \mu} \log Z_t(s) \geq 0 .
\]
\end{lemma}

\begin{proof}
First, let us show that $\log Z_t(s)\geq 0$. To see this, observe:
\begin{multline*}
    \log Z_t(s) = \log \sum_a \pi^{(t)}(a| s)\exp(\eta A^{(t)}(s,a)/(1-\gamma))\\
    \geq \sum_a \pi^{(t)}(a| s) \log \exp(\eta A^{(t)}(s,a)/(1-\gamma))
    = \frac{\eta}{1-\gamma}\sum_a \pi^{(t)}(a| s) A^{(t)}(s,a) = 0.
\end{multline*}
where we have used Jensen's inequality
on the concave function $\log x$ and that $\sum_a
\pi^{(t)}(a| s)A^{(t)}(s,a) = 0$.
By the performance
difference lemma,
\begin{align*}
    V^{(t+1)}(\mu) - V^{(t)}(\mu) &= \frac{1}{1-\gamma} \E_{s\sim d^{(t+1)}_{\mu}} \sum_a \pi^{(t+1)}(a| s) A^{(t)}(s,a)\\
    &= \frac{1}{\eta} \E_{s\sim d^{(t+1)}_{\mu}} \sum_a \pi^{(t+1)}(a| s) \log \frac{\pi^{(t+1)}(a| s)Z_t(s)}{\pi^{(t)}(a| s)}\\
    &= \frac{1}{\eta} \E_{s\sim d^{(t+1)}_{\mu}}
      \kl(\pi^{(t+1)}_s||\pi^{(t)}_s) + \frac{1}{\eta} \E_{s\sim
      d^{(t+1)}_{\mu}} \log Z_t(s)\\
&\geq \frac{1}{\eta} \E_{s\sim d^{(t+1)}_{\mu}} \log Z_t(s)
\geq \frac{1-\gamma}{\eta} \E_{s\sim \mu} \log Z_t(s),
\end{align*}
where the last step uses that $d^{(t+1)}_{\mu} \geq (1-\gamma)
\mu$ (by~\eqref{eqn:dpi}) and that $\log Z_t(s)\geq 0$.
\end{proof}

With this lemma, we now prove Theorem~\ref{thm:npg}.

\begin{proof}[\textbf{of Theorem~\ref{thm:npg}}]
Since $\rho$ is fixed, we use $d^\star$ as shorthand for
$d^{\pi^\star}_{\rho}$; we also use $\pi_s$ as shorthand for the vector of
$\pi(\cdot| s)$.
By the performance difference lemma (Lemma~\ref{lemma:perf_diff}),
\begin{align*}
  V^{\pi^\star}(\rho) &- V^{(t)}(\rho)
= \frac{1}{1-\gamma} \E_{s\sim d^{\star}} \sum_a \pi^\star(a| s) A^{(t)}(s,a)\\
&= \frac{1}{\eta} \E_{s\sim d^{\star}} \sum_a \pi^\star(a| s) \log \frac{\pi^{(t+1)}(a| s) Z_t(s)}{\pi^{(t)}(a| s)}\\
&= \frac{1}{\eta} \E_{s\sim d^{\star}} \left(\kl(\pi^\star_s ||
\pi^{(t)}_s) - \kl(\pi^\star_s || \pi^{(t+1)}_s) + \sum_a \pi^*(a| s) \log Z_t(s)\right) \\
&= \frac{1}{\eta} \E_{s\sim d^{\star}} \left(\kl(\pi^\star_s ||
\pi^{(t)}_s) - \kl(\pi^\star_s || \pi^{(t+1)}_s) + \log Z_t(s)\right),
\end{align*}
where we have used the closed form of our updates from
Lemma~\ref{lemma:npg-softmax} in the second step.
%, we can substitute for $A^{(t)}(s,a)$ in the above equality in terms
%of the successive policies to obtain Since $V^{(t+1)}(s_0) \geq
%V^{(t)}(s_0)$ under the NPG update, due to
%Lemma~\ref{lemma:npg-gap}, we see that

By applying Lemma~\ref{lemma:npg-gap} with $d^{\star}$ as the
starting state distribution, we have:
\[
\frac{1}{\eta} \E_{s\sim d^{\star}} \log Z_t(s) \leq \frac{1}{1-\gamma}
\Big(V^{(t+1)}(d^{\star}) - V^{(t)}(d^{\star})\Big)
\]
which gives us a bound on $\E_{s\sim d^{\star}} \log Z_t(s)$.

Using the above equation and that $V^{(t+1)}(\rho) \geq V^{(t)}(\rho)$ (as $V^{(t+1)}(s) \geq V^{(t)}(s)$ for all states $s$ by
Lemma~\ref{lemma:npg-gap}), we have:
\begin{align*}
%V^{\pi^\star}(\rho) - V^{(T-1)}(\rho) &\leq V^{\pi^\star}(\rho) - V^{(T-1)}(\rho)\\
V^{\pi^\star}(\rho) &- V^{(T-1)}(\rho)
\leq \frac{1}{T} \sum_{t=0}^{T-1} (V^{\pi^\star}(\rho) - V^{(t)}(\rho)) \\
&= \frac{1}{\eta T} \sum_{t=0}^{T-1} \E_{s\sim d^{\star}}
(\kl(\pi^\star_s || \pi^{(t)}_s) - \kl(\pi^\star_s || \pi^{(t+1)}_s))
+ \frac{1}{\eta T} \sum_{t=0}^{T-1} \E_{s\sim d^{\star}} \log Z_t(s) \\
&\leq \frac{\E_{s\sim d^{\star}} \kl(\pi^\star_s||\pi^{(0)})}{\eta T}
+ \frac{1}{(1-\gamma) T} \sum_{t=0}^{T-1} \Big(V^{(t+1)}(d^{\star}) - V^{(t)}(d^{\star})\Big) \\
&= \frac{\E_{s\sim d^{\star}}
\kl(\pi^\star_s||\pi^{(0)})}{\eta T}
+ \frac{V^{(T)}(d^{\star}) - V^{(0)}(d^{\star})}{(1-\gamma)T} \\
&\leq \frac{\log |\Acal|}{\eta T} + \frac{1}{(1-\gamma)^2T}.
\end{align*}
The proof is  completed using that $V^{(T)}(\rho) \geq V^{(T-1)}(\rho)$.
\end{proof}




\section{Bibliographic Remarks and Further Readings}\label{bib:pg_opt}

The natural policy gradient method was originally presented in
\cite{Kakade01}; a number of arguments for this method have been provided based on
information geometry~\citep{Kakade01,bagnell2003covariant,Peters:2008:NA:1352927.1352986}.

The convergence rates in this chapter are largely derived
from~\cite{agarwal2019optimality}. The proof strategy for the NPG analysis has origins in the online regret
framework in changing MDPs~\citep{even-dar2009online}, which would
result in a worst rate in comparison
to~\cite{agarwal2019optimality}.  This observation that the proof
strategy from ~\citep{even-dar2009online} provided  a convergence rate
for the NPG was made in~\cite{DBLP:journals/corr/NeuJG17}.
The faster NPG rate we present here is due to~\cite{agarwal2019optimality}.
The analysis of the MD-MPI algorithm~\citep{geist2019theory} also
implies a $O(1/T)$ rate for the NPG, though with worse dependencies on
other parameters.

Building on ideas
in~\cite{agarwal2019optimality}, \cite{mei2020global} showed that, for
the softmax policy class, both the gradient ascent and entropy
regularized gradient ascent asymptotically converge at a $O(1/t)$; it
is unlikely these methods are have finite rate which are polynomial in
other quantities (such as the $|\Scal|$, $|\Acal|$, $1/(1-\gamma)$, and the
  distribution mismatch coefficient).


\cite{mnih2016asynchronous} introduces the entropy regularizer (also
see~\cite{ahmed2019understanding} for a more detailed empirical
investigation).
