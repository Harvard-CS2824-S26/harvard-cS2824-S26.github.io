\chapter{Strategic Exploration in Tabular MDPs}
\label{chap:tabular_exploration}

%\newcommand*{\one}{{\textbf 1}}

We now turn to how an agent acting in an unknown MDP can obtain a
near-optimal reward over time. Compared with the previous
setting with access to a generative model, we no longer have easy
access to transitions at each state, but only have the ability to
execute trajectories in the MDP. The main complexity this adds to the
learning process is that the agent has to engage in exploration, that
is, plan to reach new states where enough samples have not been seen
yet, so that optimal behavior in those states can be learned.  

We will work with finite-horizon MDPs with a fixed start state $s_0$ (see
Section~\ref{section:finite_horizon}), and we will assume the agent learns
in an episodic setting, as introduced in
Chapter~\ref{chap:generative}. Here, in
in every episode $k$, the learner acts for $H$ step starting from a
fixed starting state $s_0$ and, at the end of the $H$-length
episode, the state is reset to $s_0$. It is straightforward to extend this
setting where the starting state is sampled from a distribution, i.e.
$s_0\sim \mu$. 

The goal of the agent is to minimize her expected cumulative regret over $K$ episodes: 
\begin{align*}
    \textrm{Regret} := \mathbb{E}\left[ K V^{\star}(s_0)  - \sum_{k=0}^{K-1}
  \sum_{h=0}^{H-1} r(s^k_h,a_h^k)  \right],
\end{align*} 
where the expectation is with respect to the randomness of the MDP
environment and, possibly,  any randomness of the agent's strategy. 

This chapter considers tabular MDPs, where $\Scal$ and $\Acal$ are discrete.
We will now present a (nearly) sublinear regret algorithm,
UCB-Value Iteration. This chapter follows the proof in
~\cite{pmlr-v70-azar17a}. We first give a simplified analysis of UCB-VI that gives a regret bound in the order of $H^2 S\sqrt{AK}$, followed by a more refined analysis that gives a $H^2 \sqrt{SAK} + H^3 S^2 A$ regret bound.   

%\sk{We don't use the $\mu$ in this chapter, right? Also can we just
%  always write $V^{\pi}(s_0)$ and not $V^{\pi}$, as let's try to keep
%  notation consistent.}
%We denote $V^{\pi}(s_0)$ as the expected total reward of $\pi$, i.e., $V^{\pi}(s_0):= \EE_{s_0\sim \mu} V^{\pi}(s_0)$.

\section{On The Need for Strategic Exploration}

%To be added...

%\sk{we should give the 'bad' example where any non-adaptive
%  information gathering strategy requires $O(|\Acal|^H)$ to hit the goal state.}


\begin{figure}
		\centering

			\begin{tikzpicture}[->,>=stealth,node distance=2.0cm]
		\node[state] (s0) {$s_0$};
		\node[state, right of=s0] (s1) {$s_1$};
		\node[state, draw=none] (d1) [right=of s1, xshift=-2.0cm] {$\cdots$};
		\node[state, right of=d1] (sh) {$s_{H}$};
		\node[state, accepting, right of=sh] (shp) {$s_{H+1}$};
		\draw (shp) edge[loop above] node{$a_1$} (shp)
		(s0) edge[bend right=30, below] node{$a_1$} (s1)
		(d1) edge[bend right=30, below] node{$a_1$} (sh)
		(sh) edge[bend right=0, below] node{$a_1$} (shp)
		(s1) edge[bend right=0, above] node{$a_2$} (s0)
		(sh) edge[bend right=0, above] node{$a_2$} (d1)
		(s1) edge[bend right=40, above] node{$a_3$} (s0)
		(s1) edge[bend right=90, above] node{$a_4$} (s0)
		(sh) edge[bend right=40, above] node{$a_3$} (d1)
		(sh) edge[bend right=90, above] node{$a_4$} (d1);
			\end{tikzpicture}
			\caption{A deterministic, chain MDP of length $H+2$. Rewards
                          are $0$ everywhere other than $r(s_{H+1},
                          a_1) = 1$. In this example we have $A = 4$.}
			\label{fig:chain_example}

	\end{figure}

Consider the chain MDP of length $H+2$ shown in
Figure~\ref{fig:chain}. The starting state of interest is state $s_0$.  We consider a uniformly random policy $\pi_h(s)$ that at any time step $h$ and any state $s$, selects an action from all A many actions uniform randomly. It is easy to see that starting from $s_0$, the probability of such uniform random policy hitting the rewarding state $s_{H+1}$ is exponentially small, i.e., $O\left( A^{H}\right)$, since it needs to select the right action that moves one step right at every time step h.
This example demonstrates that we need to perform strategic exploration in order to avoid the exponential sample complexity. %Indeed, in this chapter, we give an algorithm UCB-VI that can find an $\epsilon$ near-optimal policy with number of time steps scaling polynomially with respect to $S,A, H, 1/\epsilon$. 



\section{The UCB-VI algorithm}

If the learner then executes
$\pi^k$ in the underlying MDP to generate a single trajectory $\tau^k
= \{s^k_h,a_h^k\}_{h=0}^{H-1}$ with $a_h = \pi_h^k(s_h^k)$ and
$s_{h+1}^k \sim P_h(\cdot |s_h^k,a_h^k)$. 
We first define some notations below. Consider the very beginning of episode $k$. We use the history information up to the end of episode $k-1$ (denoted as $\Hcal_{<k}$) to form some statistics. Specifically, we define:
\begin{align*}
    & N_h^k(s,a,s') = \sum_{i=0}^{k-1} \one\{(s_h^i,a_h^i, s^i_{h+1}) = (s,a,s')\}, \\
    & N_h^k(s,a) = \sum_{i=0}^{k-1} \one\{(s_h^i,a_h^i) = (s,a)\}, \forall h, s,a.
\end{align*} Namely, we maintain counts of how many times $s,a,s'$ and $s,a$ are visited at time step $h$ from the beginning of the learning process to the end of the episode $k-1$. We use these statistics to form an empirical model:
\begin{align}
    \widehat{P}^k_h(s'|s,a) = \frac{N^k_h(s,a,s')}{N^k_h(s,a)}, \forall h, s, a,s'.\label{eq:empirical_model}
\end{align}


\begin{algorithm}[!t]
\begin{algorithmic}[1]
  \Require reward function $r$ (assumed to be known), confidence parameters
\For{$k = 0\dots K-1 $}
    \State Compute $\widehat{P}^k_h$ as the empirical estimates, for all $h$ (Eq.~\ref{eq:empirical_model})
    \State Compute reward bonus $b^k_h$ for all h (Eq.~\ref{eq:reward_bonus})
    \State Run Value-Iteration on $\{\widehat{P}^k_h, r+b^k_h\}_{h=0}^{H-1}$ (Eq.~\ref{eq:value_iter})
    \State Set $\pi^k$ as the returned policy of VI.
\EndFor
\end{algorithmic}
\caption{UCBVI}
\label{alg: ucbvi}
\end{algorithm}


We will also use the counts to define a \emph{reward bonus}, denoted as $b_h(s,a)$ for all $h, s,a$. Denote $L:= \ln\left(SAHK/\delta\right)$ ($\delta$ as usual represents the failure probability which we will define later). We define reward bonus as follows:
\begin{align}
\label{eq:reward_bonus}
    b^k_h(s,a) = 2 H\sqrt{\frac{L}{N_h^k(s,a)}}.
\end{align} %with $c$ being some universal constant.

With reward bonus and the empirical model, the learner uses \emph{Value Iteration} on the empirical transition $\widehat{P}^k_h$ and the combined reward $r_h+b^k_h$. Starting at $H$ (note that $H$ is a fictitious extra step as an episode terminates at $H-1$), we perform dynamic programming all the way to $h = 0$:
\begin{align}
    &\widehat{V}^k_H(s) = 0, \forall s, \nonumber\\
    & \widehat{Q}^k_h(s,a) = \min\left\{r_h(s,a) + b^k_h(s,a) + \widehat{P}^k_h(\cdot | s,a)\cdot \widehat{V}^k_{h+1}, \; H\right\}, \nonumber\\
    &\widehat{V}^k_h(s) = \max_{a} \widehat{Q}^k_h(s,a), \pi^k_h(s) = \argmax_{a} \widehat{Q}^k_h(s,a), \forall h, s, a. \label{eq:value_iter}
\end{align}
Note that when using $\widehat{V}^k_{h+1}$ to compute $\widehat{Q}^k_h$, we truncate the value by $H$. This is because we know that due to the assumption that $r(s,a)\in [0,1]$, no policy's Q value will ever be larger than $H$. 

Denote $\pi^k = \{ \pi^k_0, \dots, \pi^k_{H-1}\}$. Learner then executes $\pi^k$ in the MDP to get a new trajectory $\tau^{k}$. 

UCBVI repeats the above procedure for $K$ episodes. 

\section{Analysis}

We will prove the following theorem. 
\begin{theorem}[Regret Bound of UCBVI]
UCBVI achieves the following regret bound:
\begin{align*}
    \textrm{Regret} := \EE \left[ \sum_{k=0}^{K-1} \left(V^{\star}(s_0)  - V^{\pi^k}(s_0) \right)  \right]\leq 10 H^2 S\sqrt{AK \cdot \ln(SAH^2 K^2)} = \widetilde{O}\left(H^2 S\sqrt{A K}  \right)
\end{align*}\label{thm:ucbVI}
\end{theorem}

\paragraph{Remark} While the above regret is sub-optimal, the algorithm we presented here indeed achieves a sharper bound in the leading term $\widetilde{O}( H^2 \sqrt{SA K} )$ \citep{pmlr-v70-azar17a}, which gives the tight dependency bound on $S, A, K$. In Section~\ref{sec:refined_ucb_vi}, we will present a refined analysis that indeed achieves $H^2 \sqrt{SAK}$ regret. The dependency on $H$ is not tight and tightening the dependency on H requires modifications to the reward bonus (use Bernstein inequality rather than Hoeffding's inequality for reward bonus design). 

We prove the above theorem in this section. 

We start with bounding the error from the learned model $\widehat{P}_h^k$. 
%We start with Hoeffding's inequality and union bound to bound state-action wise model error. 
\begin{lemma}[State-action wise  model error] \label{lem:model_error_ucb_vi}Fix $\delta \in (0,1)$. For all $k \in [0,\dots, K-1], s\in \Scal, a\in\Acal, h \in [0,\dots, H-1]$, with probability at least $1-\delta$, we have that for any $f: \Scal\mapsto [0, H]$:
\begin{align*}
    \left\lvert \left(\widehat{P}^k_h(\cdot | s,a) - P^\star_h(\cdot | s,a)\right)^{\top} f  \right\rvert \leq 8 H \sqrt{\frac{ S \ln( SAHK/\delta) }{N_h^k(s,a)}}.
\end{align*} \label{lem:model_error}
\end{lemma}
%\begin{proof}
%Let us consider a fixed $(s,a,h, k)$ tuple. Note that $\widehat{P}^k_h(s' | s,a) = \sum_{i=1}^{k-1} \mathbf{1}\{(s_h^i,a_h^i, s_{h+1}^i) = (s,a, s') \}  / N_h^k(s,a)$. Note that here $N_k^k(s,a)$ is indeed a random quantity, and we cannot directly apply standard concentration result here. 
%Note, conditioned $(s,a)$, the samples $\{s_{h+1}^i : s_h^i=s, a_h^i = a\}_{i=1}^{k-1}$ are i.i.d samples from $P^\star_h(\cdot | s,a)$. Thus, by Proposition~\ref{app:discrete}, we have that with probability at least $1-\delta$:
%\begin{align*}
%\left\| \widehat{P}_h^{k}(\cdot | s,a ) - P^\star_h(\cdot | s,a)  \right\|_1 \leq \sqrt{S} \left( \frac{1}{\sqrt{N_h^k(s,a)}} +  \sqrt{\frac{1}{N_h^k(s,a)} \ln(1/\delta)}  \right) \leq 2 \sqrt{ \frac{S \ln(1/\delta)}{N_h^k(s,a)} }.
%\end{align*} 
%Finally, apply a union bound over all $s\in\Scal,a\in\Acal, h\in [0,\dots, H-1], k\in [0,\dots K-1]$, and use Cauchy-Schwarz inequality, i.e., $ \left\lvert \left(\widehat{P}^k_h(\cdot | s,a) - P^\star_h(\cdot | s,a)\right)^{\top} f  \right\rvert \leq \| \widehat{P}^k_h(\cdot | s,a) - P^\star_h(\cdot | s,a) \|_1 \| f \|_\infty$, we can conclude the proof. 
%\end{proof}
\paragraph{Remark} The proof of the above lemma requires a careful argument using Martingale difference sequence mainly because $N_h^k(s,a)$ itself is a random quantity.  We give a proof the above lemma at the end of this section (Sec.~\ref{sec:proof_of_l1_model_error}).
Using the above lemma will result a sub-optimal regret bound in terms of the dependence on $S$. In Section~\ref{sec:refined_ucb_vi}, we will give a different approach to bound $\left\lvert\left( \widehat{P}^k_h(\cdot | s,a) - P_h^\star(\cdot | s,a) \right)^{\top} f\right\rvert$ which leads to an improved regret bound.



%The proof of the above lemma uses Proposition~\ref{app:discrete} and a union bound over all $s,a,n,h$.
%We have seen exactly the same bound in the generative model lecture. 

The following lemma is still about model error, but this time we consider an average model error with respect to $V^\star$ --- a deterministic quantity.
\begin{lemma}[State-action wise average model error under $V^\star$] \label{lem:model_error_v_star_ucb_vi}Fix $\delta \in (0,1)$. For all $k \in [1,\dots, K-1], s\in \Scal, a\in\Acal, h \in [0,\dots, H-1]$, and consider $V^\star_h: \Scal\to [0,H]$, with probability at least $1-\delta$, we have:
\begin{align*}
    \left\lvert \widehat{P}^k_h(\cdot | s,a) \cdot V^\star_{h+1} - P^\star_h(\cdot | s,a) \cdot V^\star_{h+1} \right\rvert  \leq 2H \sqrt{\frac{ \ln( SAHN/\delta) }{N_h^k(s,a)}}.
\end{align*} \label{lem:model_error_avg}
\end{lemma}
Note that we can use Lemma~\ref{lem:model_error_ucb_vi} to bound $ \left\lvert \widehat{P}^k_h(\cdot | s,a) \cdot V^\star_{h+1} - P^\star_h(\cdot | s,a) \cdot V^\star_{h+1} \right\rvert $. However, the above lemma shows that by leveraging the fact that $V^\star$ is a deterministic quantity (i.e., independent of the data collected during learning), we can get a tighter upper bound which does not scale polynomially with respect to $S$. 
%\textbf{Please complete the proof of Lemma~\ref{lem:model_error_avg}}.

\begin{proof}
Consider a fixed tuple $s,a,k, h$ first. By the definition of $\widehat{P}^k_h$, we have:
\begin{align*}
\widehat{P}^k_h(\cdot | s,a) \cdot V^\star_{h+1} = \frac{1}{N_h^k(s,a)} \sum_{i=0}^{k-1} \one\{(s_h^i,a_h^i) = (s,a)\} V^{\star}_{h+1}(s_{h+1}^i).
\end{align*} 
Now denote $\Hcal_{h,i}$ as the entire history from $t = 0$ to iteration $t = i$ where in iteration $i$, $\Hcal_{h,i}$  includes history from time step $0$ up to and including time step $h$.
 We define random variables: $X_i = \one\{(s_h^i,a_h^i) = (s,a)\} V^\star_{h+1}(s_{h+1}^i) -  \mathbb{E}\left[ \one\{(s_h^i,a_h^i) = (s,a)\} V^\star_{h+1}(s_{h+1}') | \Hcal_{h,i}\right]$ for $i = 0,\dots, K-1$. These random variables have the following properties. First $| X_i | \leq H$ if $ \one\{(s_h^i,a_h^i) = (s,a)\} = 1$, else $|X_i| = 0$, which implies that $\sum_{i=0}^{k} |X_i|^2 \leq N_h^k(s,a)$.  Also we have $\mathbb{E}\left[ X_i | \Hcal_{h,i}\right] = 0$ for all $i$. Thus, $\{X_i\}_{i \geq 0}$ is a Martingale difference sequence.  Using Azuma-Hoeffding's inequality, we get that for any fixed $k$, with probability at least $1-\delta$, 
 \begin{align*}
 \left\lvert \sum_{i=0}^{k-1} X_i \right\rvert = \left\lvert \sum_{i=0}^{k-1} \one\{(s_h^i,a_h^i) = (s,a)\}V_{h+1}^\star(s_{h+1}^i)- N_h^k(s,a) \EE_{s'\sim P^\star_h(s,a)} V^\star(s') \right\rvert \leq 2H\sqrt{ N_h^k(s,a) \ln(1/\delta)}.
 \end{align*}
Divide $N_h^k(s,a)$ on both sides and use the definition of $\widehat{P}_h(\cdot | s,a)^{\top} V^\star_{h+1}$, we get:
\begin{align*}
\left\lvert \widehat{P}_h(\cdot | s,a)^{\top} V^\star_{h+1} - P_h^\star(\cdot|s,a)^{\top} V^\star_{h+1}  \right\rvert \leq 2H\sqrt{  \ln(1/\delta) / N_h^k(s,a)}.
\end{align*} 
%Note that for any $(s_h^i, a_h^i) = (s,a)$, we have $\mathbb{E}\left[ V^\star_{h+1}(s^i_{h+1}) | s_h^i,a_h^i   \right]  = P^\star_h(\cdot | s,a) \cdot V^\star_{h+1}$. Also note that $\left\lvert V^\star(s_{h+1}^i)\right\rvert \leq H$ for all $h,i$. 
%Thus, we can apply Azuma-Hoeffding's inequality here to bound $\left\lvert \widehat{P}^k_h(\cdot | s,a) \cdot V^\star_{h+1} - P_h(\cdot|s,a)\cdot V^\star_{h+1}  \right\rvert$, i.e., with probability at least $1-\delta$, for this fixed tuple  $s,a,k, h$, we have:
%\begin{align*}
%\left\lvert \widehat{P}^k_h(\cdot | s,a) \cdot V^\star_{h+1} - P^\star_h(\cdot | s,a) \cdot V^\star_{h+1} \right\rvert \leq 2H \sqrt{\frac{ \ln( 1/\delta) }{N_h^k(s,a)}}
%\end{align*}


 With a union bound over all $s\in\Scal,a\in\Acal,k \in [N], h \in [H]$, we conclude the proof. 
\end{proof}


We denote the two inequalities in Lemma~\ref{lem:model_error} and Lemma~\ref{lem:model_error_avg} as event $\mathcal{E}_{model}$. Note that the failure probability of $\mathcal{E}_{model}$ is at most $2\delta$. Below we  condition on $\mathcal{E}_{model}$ being true (we deal with failure event at the very end). 



%The above lemma is also used in the generative model lecture (recall we have $V^\star$ as $f$ here). The key here is that $f$ (and $V^\star$) needs to be independent of the data that used to form $\widehat{P}^n_h$. This obviously is true for $V^\star$ which is a deterministic quantity that exists at the moment when the MDP is specified. 


Now we study the effect of reward bonus. Similar to the idea in multi-armed bandits, we want to pick a policy $\pi^k$, such that the value of $\pi^k$ in under the combined reward $r_h+b_h^k$ and the empirical model $\widehat{P}^k_h$ is optimistic, i.e., we want $\widehat{V}^k_0(s_0) \geq V^\star_0(s_0)$ for all $s_0$. The following lemma shows that via reward bonus, we are able to achieve this optimism. 
\begin{lemma}[Optimism]\label{lem:optimism_ucb_vi}
Assume $\mathcal{E}_{model}$ is true. For all episode $k$, we have:
\begin{align*}
    \widehat{V}^k_0(s_0) \geq V_0^\star(s_0), \forall s_0\in \Scal;
\end{align*} where $\widehat{V}^k_h$ is computed based on VI in Eq.~\ref{eq:value_iter}.
\end{lemma}
\begin{proof}
We prove via induction. At the additional time step $H$ we have $\widehat{V}^k_H(s) = V^\star_H(s) = 0$ for all $s$. %Below we provide a proof sketch. 

Starting at $h+1$, and assuming we have $\widehat{V}^k_{h+1}(s) \geq V^\star_{h+1}(s)$ for all $s$, we move to $h$ below. 

Consider any $s,a\in\Scal\times\Acal$.
First, if $Q^k_h(s,a) = H$, then we have $Q^k_h(s,a) \geq Q^\star_h(s,a)$. 
\begin{align*}
    \widehat{Q}^k_h(s,a) - Q_h^\star(s,a)& = b_h^k(s,a) + \widehat{P}^k_{h}(\cdot|s,a)\cdot \widehat{V}^k_{h+1} -  {P}^\star_{h}(\cdot|s,a)\cdot {V}^\star_{h+1} \\
    & \geq b_h^k(s,a) + \widehat{P}^k_{h}(\cdot|s,a)\cdot {V}^\star_{h+1} -  {P}^\star_{h}(\cdot|s,a)\cdot {V}^\star_{h+1} \\
    & = b_h^k(s,a) + \left( \widehat{P}_h^k(\cdot|s,a) - P^\star_h(\cdot|s,a) \right)\cdot V^\star_{h+1} \\
    & \geq b^k_h(s,a) -  2H \sqrt{\frac{ \ln(SAHK/\delta)}{N_h^k(s,a)}} \geq 0. 
\end{align*} where the first inequality is from the inductive hypothesis, and the last inequality uses  Lemma~\ref{lem:model_error_avg} and the definition of bonus.

From $\widehat{Q}^k_{h}$, one can finish the proof by showing $\widehat{V}^k_h(s) \geq V^\star_h(s), \forall s$.
\end{proof}

Before we prove our final result, one more technical lemma will be
helpful:

\begin{lemma}\label{lem:potential}
Consider arbitrary $K$ sequence of trajectories $\tau^k = \{s_h^k,a_h^k\}_{h=0}^{H-1}$ for $k = 0,\dots,K-1$. We have
\begin{align*}
    \sum_{k=0}^{K-1} \sum_{h=0}^{H-1} \frac{1}{\sqrt{N_h^k(s_h^k,a_h^k)}} \leq 2 H \sqrt{SA K}. 
\end{align*}
\end{lemma}
\begin{proof}
We swap the order of the two summation above:
\begin{align*}
& \sum_{k=0}^{K-1} \sum_{h=0}^{H-1} \frac{1}{\sqrt{N_h^k(s_h^k,a_h^k)}}  =  \sum_{h=0}^{H-1} \sum_{k=0}^{K-1}  \frac{1}{\sqrt{N_h^k(s_h^k,a_h^k)}}  = \sum_{h=0}^{H-1} \sum_{s,a \in \Scal\times\Acal} \sum_{i=1}^{N_h^K(s,a)} \frac{1}{\sqrt{i}} \\
 & \leq 2\sum_{h=0}^{H-1} \sum_{s,a\in \Scal\times\Acal} \sqrt{ N_h^K(s,a)} \leq \sum_{h=0}^{H-1} \sqrt{ SA \sum_{s,a} N_h^K(s,a)} =  H \sqrt{SA K}, 
\end{align*} where in the first inequality we use the fact that $\sum_{i=1}^N 1/\sqrt{i} \leq 2 \sqrt{N}$, and in the second inequality we use CS inequality. 
\end{proof}

%Now we are almost ready to conduct the final steps. 
The proof of our main theorem now follows:

\begin{proof}[Proof of Theorem~\ref{thm:ucbVI}]

Let us consider episode $k$ and denote $\Hcal_{<k}$ as the history up to the end of episode $k-1$. We consider bounding $V^{\star} - V^{\pi^k}$.  %Note that this is a conditional expectation that conditions on history $\Hcal_{<n}$, the expectation is only with respect to the randomness in episode $n$. We have:
Using optimism and the simulation lemma, we can get the following result: 
\begin{align}\label{eq:sim_lem}
 V^{\star} (s_0)- V^{\pi^k}(s_0) \leq \widehat{V}^k_0(s_0) - V^{\pi^k}_0(s_0) \leq \sum_{h=0}^{H-1}\EE_{s_h,a_h\sim d^{\pi^k}_h}\left[ b_h^k(s_h,a_h) + \left( \widehat{P}^k_h(\cdot|s_h,a_h) - P^\star(\cdot | s_h,a_h) \right)\cdot \widehat{V}^{\pi^k}_{h+1} \right]
\end{align}
We prove the above two inequalities in the lecture. {We leave the proof of the above inequality (Eq~\ref{eq:sim_lem}) as an exercise for readers. Note that this is slightly different from the usual simulation lemma, as here we truncate $\widehat{V}$ by $H$ during VI}.


Under $\mathcal{E}_{model}$, we can bound $ \left( \widehat{P}^k_h(\cdot|s_h,a_h) - P^\star(\cdot | s_h,a_h) \right)\cdot \widehat{V}^{\pi^k}_{h+1}$ (recall Lemma~\ref{lem:model_error}) with a Holder's inequality:
\begin{align*}
     \left\lvert \left( \widehat{P}^k_h(\cdot|s_h,a_h) - P^\star(\cdot | s_h,a_h) \right)\cdot \widehat{V}^{\pi^k}_{h+1}\right\rvert  &\leq \left\|  \widehat{P}^k_h(\cdot|s_h,a_h) - P^\star(\cdot | s_h,a_h) \right\|_1 \left\| \widehat{V}^{\pi^k}_{h+1} \right\|_{\infty} \\
     & \leq 8H \sqrt{ \frac{S  L }{N_h^k(s,a)} },
\end{align*} where recall $L:= \ln(SAKH / \delta)$.
Hence, for the per-episode regret $V^{\star} (s_0)- V^{\pi^k}(s_0)$, we obtain:
\begin{align*}
    V^{\star}(s_0) - V^{\pi^k}(s_0)& \leq \sum_{h=0}^{H-1}\EE_{s_h,a_h\sim d^{\pi^k}_h}\left[ b_h^k(s_h,a_h) + 8H\sqrt{SL/N_h^k(s_h,a_h)} \right]\\
    &\leq \sum_{h=0}^{H-1}\EE_{s_h,a_h\sim d^{\pi^k}_h}\left[ 10 H \sqrt{S L/N_h^k(s_h,a_h)} \right] \\
    &= 10H\sqrt{SL}\cdot \EE\left[\sum_{h=0}^{H-1} \frac{1}{\sqrt{N_h^k(s_h^k,a_h^k)}} \big\vert \Hcal_{<k}\right],
\end{align*} where in the last term the expectation is taken with respect to the  trajectory $\{s_h^k, a_h^k\}$ (which is generated from $\pi^k$) while conditioning on all history $\Hcal_{<k}$ up to and including the end of episode $k-1$.



Now we sum all episodes together and take the failure event into consideration:
\begin{align}
    \EE\left[ \sum_{k=0}^{K-1} V^\star(s_0) - V^{\pi^k}(s_0) \right] & = \EE\left[ \one\{\mathcal{E}_{model}\} \left( \sum_{k=0}^{K-1} V^\star(s_0) - V^{\pi^k}(s_0)\right) \right] + \EE\left[\one\{\overline{\mathcal{E}}_{model}\} \left( \sum_{k=0}^{K-1} V^\star(s_0) - V^{\pi^k}(s_0)\right)\right] \nonumber \\
    &  \leq \EE\left[ \one\{\mathcal{E}_{model}\} \left( \sum_{k=0}^{K-1} V^\star(s_0) - V^{\pi^k}(s_0)\right) \right] + 2\delta KH \label{eq:expectation_regret_decomposition_ucb_vi}\\
    & \leq 10H\sqrt{S\ln(SAHK/\delta)}\EE\left[ \sum_{k=0}^{K-1} \sum_{h=0}^{H-1} \frac{1}{\sqrt{N_h^k(s_h^k,a_h^k)}}    \right] + 2\delta K H, \nonumber
    %= \sum_{n=1}^N \EE\left[ V^\star - V^{\pi^n} \vert \Hcal_{< n} \right] \\
    %&= 2H\sqrt{S\ln(SAHN/\delta)}   \EE\left[ \sum_{n=1}^N \sum_{h=0}^{H-1} \frac{1}{\sqrt{N_h^n(s_h^n,a_h^n)}} \right]
\end{align}  where the last inequality uses the law of total expectation.
%where the expectation above is the conditional expectation that takes randomness with respect to the trajectory from $\pi^n$ conditioned on history up to the end of episode $n-1$. Continue, we have:

We can bound the double summation term above using lemma~\ref{lem:potential}.
%We went through the proof in the lecture and \textbf{please complete the proof of the above lemma}
We can conclude that:
\begin{align*}
    \EE\left[ \sum_{k=0}^{K-1} V^\star(s_0) - V^{\pi^k}(s_0) \right] \leq 10 H^2 S \sqrt{AK \ln(SAHK/\delta)} + 2\delta KH. 
\end{align*} Now set $\delta = 1/KH$, we get:
\begin{align*}
   \EE\left[ \sum_{k=0}^{K-1} V^\star(s_0) - V^{\pi^k}(s_0) \right] \leq 10 H^2 S \sqrt{ AK \ln(SAH^2K^2) } + 2 = O\left( H^2 S\sqrt{ AK \ln(SAH^2K^2)}\right).
\end{align*}

This concludes the proof of Theorem~\ref{thm:ucbVI}.
\end{proof}

\subsection{Proof of Lemma~\ref{lem:model_error}}
\label{sec:proof_of_l1_model_error}

Here we include a proof of Lemma~\ref{lem:model_error}. The proof consists of two steps, first we using Martingale concentration to bound $| (\widehat{P}_h^k(\cdot | s,a) - P^\star_h(\cdot |s,a))^{\top} f |$ for all $s,a,h, k$, but for a fixed $f: \Scal\mapsto [0,H]$. Note that there are infinitely many such $f$ that maps from $\Scal$ to $[0,H]$. Thus, the second step is to use a covering argument (i.e., $\epsilon$-net on all $\{f:\Scal\mapsto[0,H]\}$) to show that $| (\widehat{P}_h^k(\cdot | s,a) - P^\star_h(\cdot |s,a))^{\top} f |$ is bounded for all $s,a,h,k$, and all $f:\Scal\mapsto [0,H]$.


\begin{proof}[Proof of Lemma~\ref{lem:model_error}]
We consider a fixed tuple $s,a, k, h, f$ first. Recall the definition of $\widehat{P}_h^k(s,a)$, and we have:
$\widehat{P}_h^k(\cdot | s,a)^{\top} f = \sum_{i=0}^{k-1} \one\{s_h^i,a_h^i = s,a\} f(s_{h+1}^i) / N_h^k(s,a)$.

Define $\Hcal_{h,i}$ as the history starting from the beginning of iteration $0$ all the way up to and include time step $h$ at iteration $i$. Define random variables $X_i$ as $X_i =  \one\{s_h^i,a_h^i = s,a\} f(s_{h+1}^i)  -  \one\{s_h^i,a_h^i = s,a\} \EE_{s'\sim P^\star_h(s,a)}f(s')$. We now show that $\{X_i\}$ is a Martingale difference sequence. First we have $\EE[X_i | \Hcal_{h,i}] = 0$, since $\one\{s_h^i,a_h^i = s,a\}$ is a deterministic quantity given $\Hcal_{h,i}$. Second, we have $|X_i| = 0$ for $(s_h^i,a_h^i) \neq (s,a)$, and $|X_i| \leq H$ when $(s_h^i,a_h^i) = (s,a)$. Thus, we have shown that it is a Martingale difference sequence. 

Now we can apply Azuma-Hoeffding's inequality (Theorem~\ref{thm:Azuma}). With probability at least $1-\delta$, we have:
\begin{align*}
\left\lvert \sum_{i=0}^{k-1} X_i \right\rvert = \left\lvert \sum_{i=0}^{k-1} \one\{(s_h^i,a_h^i) = (s,a)\} f(s^i_{h+1}) - N_h^k(s,a) \EE_{s'\sim P_h^\star(s,a)} f(s')  \right\rvert \leq 2H \sqrt{ N_h^k(s,a) \ln(1/\delta)  }.
\end{align*} 
Now apply union bound over all $s\in\Scal, a\in\Acal, h\in [0,\dots, H-1], k \in [0, \dots, K-1]$, we have that with probability at least $1-\delta$, for all $s,a,h,k$, we have:
\begin{align}
 \left\lvert \sum_{i=0}^{k-1} \one\{(s_h^i,a_h^i) = (s,a)\} f(s^i_{h+1}) - N_h^k(s,a) \EE_{s'\sim P_h^\star(s,a)} f(s')  \right\rvert \leq 2H \sqrt{ N_h^k(s,a) \ln(SAKH/\delta)  }.
\label{eq:convergence_ucb_vi_before_f}
\end{align} 
Note that the above result only holds for the fixed $f$ that we specified at the beginning of the proof. We are left to perform a covering argument over all functions that map from $\Scal$ to $[0,H]$. We use a standard $\epsilon$-net argument for that.  We abuse notation a bit and denote $f$ as a vector in $[0,H]^{S}$. 

Note that $\|f\|_2 \leq H \sqrt{S}$ for any $f:\Scal\mapsto[0,H]$. A standard $\epsilon$-net argument shows that we can construct an $\epsilon$-net $\Ncal_{\epsilon}$ over $[0,H]^S$ with $|\Ncal_{\epsilon}| \leq (1 + 2 H\sqrt{S} / \epsilon)^S$, such that for any $f\in [0,H]^S$, there exists a point $f' \in \Ncal_\epsilon$, such that $\|f - f'\|_2 \leq \epsilon$. Note that the $\epsilon$-net $\Ncal_\epsilon$ is independent of the training data during the learning process.

Now using Eq.~\ref{eq:convergence_ucb_vi_before_f} and a union bound over all $f$ in $\Ncal_\epsilon$, we have that with probability at least $1-\delta$, for all $s,a,k,h,f\in\Ncal_\epsilon$, we have:
\begin{align*}
\left\lvert \frac{\sum_{i=0}^{k-1} \one\{(s_h^i,a_h^i) = (s,a)\} f(s^i_{h+1})}{N_h^k(s,a)} - \EE_{s'\sim P_h^\star(s,a)} f(s')  \right\rvert \leq 2H \sqrt{\frac{ S\ln(SAKH(1+2H\sqrt{S} / \epsilon)/\delta)  }{ N_h^k(s,a)}}.
\end{align*} 

Finally, for any $f\in[0,H]^S$, denote its closest point in $\Ncal_\epsilon$ as $f'$, we have:
\begin{align*}
&\left\lvert \frac{\sum_{i=0}^{k-1} \one\{(s_h^i,a_h^i) = (s,a)\} f(s^i_{h+1})}{N_h^k(s,a)} - \EE_{s'\sim P_h^\star(s,a)} f(s')  \right\rvert \\
& \leq  \left\lvert \frac{\sum_{i=0}^{k-1} \one\{(s_h^i,a_h^i) = (s,a)\} f'(s^i_{h+1})}{N_h^k(s,a)} - \EE_{s'\sim P_h^\star(s,a)} f'(s')  \right\rvert + \left\lvert  \frac{\sum_{i=0}^{k-1} \one\{(s_h^i,a_h^i) = (s,a)\} (f(s^i_{h+1}) - f'(s^i_{h+1}))}{N_h^k(s,a)}    \right\rvert \\
& \qquad + \left\lvert \EE_{s'\sim P_h^\star(s,a)} (f(s') - f'(s'))  \right\rvert \\
& \leq  2H \sqrt{\frac{ S\ln\left(SAKH(1+2H\sqrt{S}/\epsilon)/\delta\right)  }{ N_h^k(s,a)}} + 2 \epsilon,
\end{align*} where in the last inequality we use the fact that if $\|f-f'\|_2 \leq \epsilon$, then $|f(s) - f'(s)| \leq \epsilon,\forall s\in \Scal$. 

Now we can set $\epsilon = 1/{K}$ and use the fact that $N_h^k(s,a) \leq K$, we have:
\begin{align*}
&\left\lvert \frac{\sum_{i=0}^{k-1} \one\{(s_h^i,a_h^i) = (s,a)\} f(s^i_{h+1})}{N_h^k(s,a)} - \EE_{s'\sim P_h^\star(s,a)} f(s')  \right\rvert & \leq  2H \sqrt{\frac{ S\ln\left(SAK H(1+2HK\sqrt{S})/\delta\right)  }{ N_h^k(s,a)}} + \frac{2}{K}\\
& \leq 4H \sqrt{\frac{ S\ln( 4H^2S^2K^2 A /\delta)  }{ N_h^k(s,a)}} \leq 8H \sqrt{\frac{ S \ln(HSKA/\delta) }{N_h^k(s,a)}},
\end{align*} where the last inequality uses the simple inequality $\ln(4H^2 S^2 K^2 A) \leq 4\ln(HSKA)$.
This concludes the proof.
\end{proof}



\section{An Improved Regret Bound}
\label{sec:refined_ucb_vi}

In this section, we show that UCB-VI also enjoys the regret bound $\tilde{O}\left( H^2 \sqrt{SAK} + H^3 S^2 A \right)$.
\begin{theorem}[Improved Regret Bound for UCB-VI] \label{thm:improved_regret_ucb_vi}The total expected regret is bounded as follows:
\begin{align*}
&\mathbb{E}\left[ KV^\star(s_0) - \sum_{k=0}^{K-1} V^{\pi^k}(s_0)  \right] \\
& \qquad \leq  8 e H^2 \sqrt{SAK}\cdot \sqrt{ \ln(SAH^2K^2)} + 2 e H^3 S^2A \cdot \ln(K) \ln(SAH^2 K^2) = \tilde{O}\left( H^2 \sqrt{SAK} + H^3 S^2 A \right).
\end{align*}
\end{theorem}


Note that comparing to the regret bound we had in the previous section, this new regret improves the $S$ dependence in the leading order term (i.e., the term that contains $\sqrt{K}$), but at the cost of introducing a lower order term $H^3 S^2 A$. 

We  first give an improved version of the model error presented in Lemma~\ref{lem:model_error_ucb_vi}.
\begin{lemma} \label{lem:tighter_model_error_ucbvi}With probability at least $1-\delta$, for all $h, k, s,a$, and all $f:\Scal\mapsto [0, H]$, we have:
\begin{align*}
\left\lvert \left( \widehat{P}^k_h(\cdot | s,a) - P^\star_h(\cdot | s,a)  \right)^{\top} f \right\rvert \leq  \frac{ \mathbb{E}_{s'\sim P^\star_h(\cdot | s,a)} f(s')   }{H} +  \frac{3 H^2 SL }{N_h^k(s,a)}.
\end{align*}
\end{lemma}

Below we prove this lemma. Recall the definition of $\widehat{P}_h^k(s' | s,a)$ in Eq.~\ref{eq:empirical_model}.
%\begin{align*}
%\widehat{P}^k_h(s' | s,a) = \sum_{i=0}^{k-1} \one\{(s_h^i,a_h^i, {s_{h+1}^i}) = (s,a,s')\} / N_h^k(s,a).
%\end{align*} 
We can use Azume-Bernstein's inequality to bound $| \widehat{P}_h^k(s'| s,a) - P^\star_h(s' | s,a) |$. 
\begin{lemma}\label{lem:point_wise_model_error_ucb_vi}With probability at least $1-\delta$, for all $s,a,k, h$, we have:
\begin{align*}
\left\lvert \widehat{P}_h^k(s'| s,a) - P^\star_h(s' | s,a) \right\rvert \leq \sqrt{\frac{ 2P^\star_h(s' | s,a) L}{ N_h^k(s,a)}} + \frac{2 L }{N_h^k(s,a)},
\end{align*} where $L:= \ln(SAKH / \delta)$.
\end{lemma}
\begin{proof} Consider a fixed tuple $(s,a,h,k)$.
Define $\Hcal_{h,i}$ as the history starting from the beginning of iteration $0$ to time step $h$ at iteration $i$ (including step $h$, i.e., up to $s^i_h,a^i_h$). Define random variables $\{X_i\}_{i \geq 0}$ as $X_i = \one\{s_h^i,a_h^i, {s^i_{h+1}} = s,a,s'\} - \one\{s_h^i,a_h^i = s,a\} P^\star_h(s' | s,a)$. We now show that $\{X_i\}$ is a Martingale difference sequence. 
Note that $\EE\left[ X_i | \Hcal_{h,i} \right] = 0$. We have $|X_i| \leq 1$. To use Azuma-Bernstein's inequality, 
 we note that $\EE[ X_i^2 | \Hcal_{h,i}]$ is bounded as:
\begin{align*}
\EE[X_i^2 | \Hcal_{h,i}] =  \one\{s_h^i,a_h^i = s,a\} P^\star_h(s' | s,a) (1-P_h^\star(s' | s,a)),
\end{align*} where we use the fact that the variance of a Bernoulli with parameter $p$ is $p(1-p)$.  This means that $\sum_{i=0}^{k-1} \EE[X_i^2 | \Hcal_{h,i}] = P^\star_h(s' | s,a) (1-P_h^\star(s' | s,a)) N_h^k(s,a)$.  Now apply Bernstein's inequality on the martingale difference sequence $\{X_i\}$, we have:
\begin{align*}
\left\lvert \sum_{i=0}^{k-1} X_i \right\rvert & \leq \sqrt{\frac{ 2P^\star_h(s' | s,a) (1-P_h^\star(s' | s,a)) \ln(1/\delta)}{ N_h^k(s,a)}} + \frac{2 \ln(1/\delta) }{N_h^k(s,a)}\\
& \leq \sqrt{\frac{ 2P^\star_h(s' | s,a)  \ln(1/\delta)}{ N_h^k(s,a)}} + \frac{2 \ln(1/\delta) }{N_h^k(s,a)}
\end{align*} Recall the definition of $L := \ln(SAHK / \delta)$. Apply a union bound over all $s\in\Scal,a\in\Acal, h\in [0,\dots, H-1], k\in [0, \dots, K-1]$, we conclude the proof. 
\end{proof}

Now we are ready to prove Lemma~\ref{lem:tighter_model_error_ucbvi}.

\begin{proof}[Proof of Lemma~\ref{lem:tighter_model_error_ucbvi}] 
Let us condition on the event in Lemma~\ref{lem:point_wise_model_error_ucb_vi} being true, of which the probability is at least $1-\delta$. 

Take any function $f: \Scal\mapsto [0, H]$, we have:
\begin{align*}
&\left\lvert \left(\widehat{P}_h^k(\cdot | s,a) - P^\star_h(\cdot | s,a)\right)^{\top} f \right\rvert \\
& \leq \sum_{s'\in\Scal} \left\lvert \widehat{P}_h^k(s'|s,a) - P^\star_h(\cdot | s,a)  \right\rvert f(s')  \leq \sum_{s'\in\Scal}  \sqrt{\frac{ 2L P^\star_h(s'|s,a) f^2(s')    }{N_h^k(s,a)}}  +  \sum_{s'\in\Scal}\frac{2 L f(s')}{N_h^k(s,a)} \\
& \leq \sum_{s'\in\Scal}  \sqrt{\frac{ 2L P^\star_h(s'|s,a) f^2(s')    }{N_h^k(s,a)}}  +  \frac{2 S H L }{N_h^k(s,a)}  \leq \sqrt{S} \sqrt{ \frac{ \sum_{s'\in\Scal} 2L P^\star_h(s' | s,a) f^2(s')  }{N_h^k(s,a)}  } + \frac{2 S H L }{N_h^k(s,a)} \\
& =  \sqrt{\frac{  2 H^2 S L  }{N_h^k(s,a)} \cdot \frac{   \sum_{s'\in\Scal} P^\star_h(s' | s,a) f^2(s')}{H^2}  } + \frac{2 S H L }{N_h^k(s,a)}  \leq   \frac{H^2 SL }{N_h^k(s,a)} + \frac{ \mathbb{E}_{s'\sim P^\star_h(\cdot | s,a)} f^2(s')   }{H^2} + \frac{2 S H L }{N_h^k(s,a)}  \\
& \leq   \frac{ \mathbb{E}_{s'\sim P^\star_h(\cdot | s,a)} f(s')   }{H} +  \frac{H^2 SL }{N_h^k(s,a)} + \frac{2 S H L }{N_h^k(s,a)},
\end{align*} where the third inequality uses that condition that $\|f\|_{\infty} \leq H$, the fourth inequality uses Cauchy-Schwarz inequality, the fifth inequality uses the inequality that $a b \leq (a^2 + b^2)/2$ for any real numbers $a$ and $b$, and the last inequality uses the condition that $\|f\|_\infty\leq H$ again. This concludes the proof.
\end{proof}

So far in Lemma~\ref{lem:tighter_model_error_ucbvi}, we have not specify what function $f$ we will use. The following lemma instantiate $f$ to be $\widehat{V}^k_{h+1} - V^{\star}_{h+1}$. Recall the bonus definition, $b_h^k(s,a) := 2H \sqrt{ L / N_h^k(s,a)}$ where $L := \ln(SAHK /\delta)$. %The intuition here is that we know from our analysis in the previous section, $\widehat{V}^k_{h+1}$ is approaching 

\begin{lemma}\label{lem:recursive_form_ucb_vi}Assume the events in Lemma~\ref{lem:tighter_model_error_ucbvi} and Lemma~\ref{lem:model_error_v_star_ucb_vi} are true.  For all $s,a,k,h$, set $f := \widehat{V}^k_{h+1} - V^{\star}_{h+1}$, we have:
\begin{align*}
\left\lvert \left( \widehat{P}^k_h(\cdot | s,a) - P^\star_h(\cdot | s,a)  \right)^{\top}  f \right\rvert  \leq  \xi_h^k(s,a) + \frac{e}{H} \mathbb{E}\left[\sum_{\tau = h+1}^{H-1} \left(\xi_{\tau}^k(s_\tau,a_\tau) + b_\tau^k(s_\tau,a_\tau)\right) \big \vert s_{h} = s,a_h = a, \pi^k \right]
\end{align*} where $\xi_\tau^i(s,a) = \frac{3H^2 SL}{N_\tau^i(s,a)}$, for any $\tau \in [0,\dots,H-1], i\in [0,\dots,K-1]$, and the expectation is with respect to the randomness from the event of following policy $\pi^k$ from $(s,a)$ starting from time step $h$.
\end{lemma}
\begin{proof}
%For notation simplicity, let us denote $\xi_h^k(s,a) = \frac{3H^2 SL}{N_h^k(s,a)}$.
Using Lemma~\ref{lem:tighter_model_error_ucbvi} and the fact that $f:=  \widehat{V}^k_{h+1} - V^{\star}_{h+1} $, we have:
\begin{align}
\label{eq:bound_on_f_ucb_vi}
\left\lvert \left( \widehat{P}^k_h(\cdot | s,a) - P^\star_h(\cdot | s,a)  \right)^{\top} f \right\rvert \leq \frac{ \EE_{s'\sim P_h^\star(\cdot | s,a)} ( \widehat{V}^k_{h+1}(s') - V^\star_{h+1}(s')) }{ H }  + \xi_h^k(s,a).
\end{align} Now consider $ \widehat{V}^k_{h+1}(s') - V^\star_{h+1}(s'))$, we have:
\begin{align*}
\widehat{V}^k_{h+1}(s') - V^\star_{h+1}(s') & \leq  \widehat{V}^k_{h+1}(s')  - \left( r(s', \pi^k(s')) + P^\star_{h+1}(\cdot | s', \pi^k(s'))^{\top} V^\star_{h+2} \right) \\
& \leq r(s', \pi^k(s')) + \widehat{P}^k_{h+1}(\cdot | s', \pi^k(s'))^{\top} \widehat{V}^k_{h+2} - \left( r(s', \pi^k(s')) + P^\star_{h+1}(\cdot | s', \pi^k(s'))^{\top} V^\star_{h+2} \right)  \\
& = \widehat{P}^k_{h+1}(\cdot | s', \pi^k(s'))^{\top} \widehat{V}^k_{h+2} - {P}^\star_{h+1}(\cdot | s', \pi^k(s'))^{\top} \widehat{V}^k_{h+2} + {P}^\star_{h+1}(\cdot | s', \pi^k(s'))^{\top} \left( \widehat{V}^k_{h+2} - V^\star_{h+2}  \right) \\
& = \left( \widehat{P}^k_{h+1}(\cdot | s', \pi^k(s')) - {P}^\star_{h+1}(\cdot | s', \pi^k(s'))\right)^{\top}\left( \widehat{V}^k_{h+2} - V^\star_{h+2} \right) \\
& \qquad +  \left( \widehat{P}^k_{h+1}(\cdot | s', \pi^k(s')) - {P}^\star_{h+1}(\cdot | s', \pi^k(s'))\right)^{\top} V^\star_{h+2}\\
& \qquad  +  {P}^\star_{h+1}(\cdot | s', \pi^k(s'))^{\top} \left( \widehat{V}^k_{h+2} - V^\star_{h+2}  \right)
\end{align*} where the first inequality uses the fact that $V^\star_{h+1}(s') = \max_{a'} (r(s',a') + P^\star_{h+1}(\cdot|s', a')^{\top} V^\star_{h+2})$, the second inequality uses the definition of $\widehat{V}^k_{h+1}$ and $\pi^k$. 
Note that the second term in the RHS of the above inequality can be upper bounded exactly by the bonus using Lemma~\ref{lem:model_error_v_star_ucb_vi}, and the first term can be further upper bounded using the same operation that we had in Lemma~\ref{lem:tighter_model_error_ucbvi}, i.e.,
\begin{align*}
 \left( \widehat{P}^k_{h+1}(\cdot | s', \pi^k(s')) - {P}^\star_{h+1}(\cdot | s', \pi^k(s'))\right)^{\top}\left( \widehat{V}^k_{h+2} - V^\star_{h+2} \right) \leq \frac{ P^\star_{h+1}(\cdot | s',\pi^k(s'))^{\top} \left( \widehat{V}^k_{h+2} - V^\star_{h+2} \right)  }{ H} + \xi_{h+1}^k(s', \pi^k(s')).
\end{align*}
Combine the above inequalities together, we arrive at:
\begin{align*}
\widehat{V}^k_{h+1}(s') - V^\star_{h+1}(s') \leq b_{h+1}^k(s',\pi^k(s')) + \xi_{h+1}^k(s',\pi^k(s')) +  \left( 1+\frac{1}{H} \right) \EE_{s''\sim P^\star_{h+1}(s',\pi^k(s'))}\left( \widehat{V}^k_{h+2}(s'') - V^\star_{h+2}(s'') \right).
\end{align*}
We can recursively apply the same operation on $\widehat{V}^k_{h+2}(s'') - V^\star_{h+2}(s'')$ till step $H$, we have:
\begin{align*}
\widehat{V}^k_{h+1}(s') - V^\star_{h+1}(s') & \leq  \mathbb{E}\left[\sum_{\tau = h+1}^{H-1} \left( 1+\frac{1}{H} \right)^{\tau - h-1} \left(\xi_{\tau}^k(s_\tau,a_\tau) + b_\tau^k(s_\tau,a_\tau)\right) \big \vert s_{h+1} = s', \pi^k \right] \\
& \leq  e \mathbb{E}\left[\sum_{\tau = h+1}^{H-1} \left(\xi_{\tau}^k(s_\tau,a_\tau) + b_\tau^k(s_\tau,a_\tau)\right) \big \vert s_{h+1} = s', \pi^k \right],
\end{align*} where the last inequality uses the fact that $(1+1/H)^x \leq e$ for any $x\in \mathbb{N}^+$. Substitute the above inequality into Eq.~\ref{eq:bound_on_f_ucb_vi}, we get:
\begin{align*}
\left\lvert \left( \widehat{P}^k_h(\cdot | s,a) - P^\star_h(\cdot | s,a)  \right)^{\top} f \right\rvert \leq \xi_h^k(s,a) + \frac{e}{H} \mathbb{E}\left[\sum_{\tau = h+1}^{H-1} \left(\xi_{\tau}^k(s_\tau,a_\tau) + b_\tau^k(s_\tau,a_\tau)\right) \big \vert s_{h} = s,a_h = a, \pi^k \right]
\end{align*}
\end{proof}

Now we are ready to derive a refined upper bound for per-episode regret. 
\begin{lemma}\label{lem:per_episode_regret_decomposition_ucb_vi}
Assume the events in Lemma~\ref{lem:recursive_form_ucb_vi} and Lemma~\ref{lem:optimism_ucb_vi} (optimism) hold. For all $k \in [0, K-1]$, we have:
\begin{align*}
V^\star(s_0) - V^{\pi^k} (s_0)\leq e \sum_{h=0}^{H-1} \EE_{s_h,a_h\sim d^{\pi^k}_h} \left[ 2b_h^k(s_h,a_h) + \xi_h^k(s_h,a_h)  \right].
\end{align*}
\end{lemma}
\begin{proof}
Via optimism in Lemma~\ref{lem:optimism_ucb_vi}, we have:
\begin{align*}
&V^{\star}(s_0) - V^{\pi^k} (s_0)\leq \widehat{V}^k_0(s_0) - V^{\pi^k}_0(s_0)  \leq \sum_{h=0}^{H-1}\EE_{s_h,a_h\sim d^{\pi^k}_h}\left[ b_h^k(s_h,a_h) + \left( \widehat{P}^k_h(\cdot|s_h,a_h) - P^\star(\cdot | s_h,a_h) \right)\cdot \widehat{V}^{k}_{h+1} \right] \\
& = \sum_{h=0}^{H-1}\EE_{s_h,a_h\sim d^{\pi^k}_h}\left[ b_h^k(s_h,a_h) + \left( \widehat{P}^k_h(\cdot|s_h,a_h) - P^\star(\cdot | s_h,a_h) \right)^{\top}\left( \widehat{V}^{k}_{h+1} - V^\star_{h+1} \right) + \left( \widehat{P}^k_h(\cdot|s_h,a_h) - P^\star(\cdot | s_h,a_h) \right)^{\top}V^\star_{h+1}\right]  \\
& \leq \sum_{h=0}^{H-1}\EE_{s_h,a_h\sim d^{\pi^k}_h}\left[ 2 b_h^k(s_h,a_h) + \left( \widehat{P}^k_h(\cdot|s_h,a_h) - P^\star(\cdot | s_h,a_h) \right)^{\top}\left( \widehat{V}^{k}_{h+1} - V^\star_{h+1} \right) \right]  \\
& \leq \sum_{h=0}^{H-1}\EE_{s_h,a_h\sim d^{\pi^k}_h}\left[ 2 b_h^k(s_h,a_h) + \xi_h^k(s_h,a_h) + \frac{e}{H} \mathbb{E}\left[\sum_{\tau = h+1}^{H-1} \left(\xi_{\tau}^k(s_\tau,a_\tau) + b_\tau^k(s_\tau,a_\tau)\right) \big \vert  \pi^k \right] \right]  \\
& \leq  e \sum_{h=0}^{H-1} \EE_{s_h,a_h\sim d^{\pi^k}_h} \left[ 2b_h^k(s_h,a_h) + \xi_h^k(s_h,a_h)  \right]
\end{align*} where the first inequality uses optimism, and the fourth inequality uses Lemma~\ref{lem:recursive_form_ucb_vi}, and the last inequality rearranges by grouping same terms together. 
\end{proof}

To conclude the final regret bound, let us denote the event $\mathcal{E}_{model}$ as the joint event in Lemma~\ref{lem:model_error_v_star_ucb_vi} and Lemma~\ref{lem:point_wise_model_error_ucb_vi}, which holds with probability at least $1-2\delta$. %Note that under the event of $\mathcal{E}_{model}$, we have 

\begin{proof}[Proof of Theorem~\ref{thm:improved_regret_ucb_vi}]
Following the derivation we have in Eq.~\ref{eq:expectation_regret_decomposition_ucb_vi}, we decompose the expected regret bound as follows:
\begin{align*}
    \EE\left[ \sum_{k=0}^{K-1} V^\star(s_0) - V^{\pi^k} (s_0)\right] & %= \EE\left[ \one\{\mathcal{E}_{model}\} \left( \sum_{k=0}^{K-1} V^\star(s_0) - V^{\pi^k}(s_0)\right) \right] + \EE\left[\one\{\overline{\mathcal{E}}_{model}\} \left( \sum_{k=0}^{K-1} V^\star(s_0) - V^{\pi^k}(s_0)\right)\right] \\
      \leq \EE\left[ \one\{\mathcal{E}_{model}\} \left( \sum_{k=0}^{K-1} V^\star(s_0) - V^{\pi^k}(s_0)\right) \right] + 2\delta KH \\
    & \leq e \EE\left[ \sum_{k=0}^{K-1} \sum_{h=0}^{H-1} \left[ 2b_h^k(s_h^k, a_h^k) + \xi_h^k(s_h^k, a_h^k)  \right]     \right] + 2\delta KH,
\end{align*}  where the last inequality uses Lemma~\ref{lem:per_episode_regret_decomposition_ucb_vi}.
Using Lemma~\ref{lem:potential}, we have that:
\begin{align*}
\sum_{k=0}^{K-1} \sum_{h=0}^{H-1} 2 b_h^k(s_h^k, a_h^k) \leq  4 H \sqrt{ L } \sum_{k}\sum_h \frac{1}{\sqrt{ N_h^k(s_h^k,a_h^k)}} \leq  8 H^2 \sqrt{SAK L}.
\end{align*}  For $\sum_k\sum_h \xi_h^k(s_h^k,a_h^k)$, we have:
\begin{align*}
\sum_{k=0}^{K-1}\sum_{h=0}^{H-1} \xi_h^k(s_h^k,a_h^k) &= 2H^2 S L\sum_{k=0}^{K-1}\sum_{h=0}^{H-1} \frac{1}{N_h^k(s_h^i,a_h^k)} = 2H^2 SL \sum_{h=0}^{H-1} \sum_{s,a} \sum_{i=1}^{N_h^{K}(s,a)} \frac{1}{i} \\
& \leq 2H^2 SA L H S \ln(K) = 2H^3 S^2A L \ln(K),
\end{align*} where in the last inequality, we use the fact that $N_h^K(s,a) \leq K$ for all $s,a,h$, and the inequality $\sum_{i=1}^{t} 1/i \leq \ln(t)$. Putting things together, and setting $\delta = 1/(KH)$, we conclude the proof. 
\end{proof}


\section{Phased $Q$-learning}

\emph{to be added}

\section{Bibliographic Remarks and Further Readings}\label{ch5:bib}

The first provably correct PAC algorithm for reinforcement learning
(which finds a near optimal policy) was due to
\citet{kearns2002optimal}, which  provided the E$^3$ algorithm; it achieves polynomial sample complexity
in tabular MDPs. \citet{brafman2002r} presents the Rmax algorithm
which provides a refined PAC analysis over
E$^3$. Both are model based approaches \cite{kakade2003sample}
improves the sample complexity to be $O(S^2A)$. Both E$^3$ and Rmax
uses the concept of absorbing MDPs to achieve optimism and balance
exploration and exploitation.  

\citet{jaksch2010optimal} provides the first $O(\sqrt(T))$ regret
bound, where $T$ is the number of timesteps in the MDP ($T$ is
proportional to $K$ in our setting); this dependence on $T$ is optimal.
Subsequently, 
\citet{pmlr-v70-azar17a,dann2017unifying} provide algorithms that, asymptotically,
achieve minimax regret bound in tabular MDPs. By this, we mean that
for sufficiently large $T$ (for $T\geq \Omega(|\Scal|^2)$), the results in
\citet{pmlr-v70-azar17a,dann2017unifying} obtain optimal dependencies
on $|\Scal|$ and $|\Acal|$. The requirement that $T\geq
\Omega(|\Scal|^2)$ before these bounds hold is essentially the
requirement that non-trivial model accuracy is required. It is an
opent question to remove this dependence.

Lower bounds are provided in~\cite{dann2015sample,osband2016on,pmlr-v70-azar17a}.

\paragraph{Further exploration strategies.}
{\bf Refs and discussion for $Q$-learning, reward free, and thompson sampling to be added... \/}

